
\begin{definition}[RectangleIntegral]\label{RectangleIntegral}\lean{RectangleIntegral}\leanok
A RectangleIntegral of a function $f$ is one over a rectangle determined by $z$ and $w$ in $\C$.
We will sometimes denote it by $\int_{z}^{w} f$. (There is also a primed version, which is $1/(2\pi i)$ times the original.)
\end{definition}


It is very convenient to define integrals along vertical lines in the complex plane, as follows.
\begin{definition}[VerticalIntegral]\label{VerticalIntegral}\leanok
Let $f$ be a function from $\mathbb{C}$ to $\mathbb{C}$, and let $\sigma$ be a real number. Then we define
$$\int_{(\sigma)}f(s)ds = \int_{\sigma-i\infty}^{\sigma+i\infty}f(s)ds.$$
\end{definition}

 We also have a version with a factor of $1/(2\pi i)$.


\begin{theorem}[existsDifferentiableOn_of_bddAbove]\label{existsDifferentiableOn_of_bddAbove}\lean{existsDifferentiableOn_of_bddAbove}\leanok
If $f$ is differentiable on a set $s$ except at $c\in s$, and $f$ is bounded above on $s\setminus\{c\}$, then there exists a differentiable function $g$ on $s$ such that $f$ and $g$ agree on $s\setminus\{c\}$.
\end{theorem}


\begin{proof}\leanok
This is the Riemann Removable Singularity Theorem, slightly rephrased from what's in Mathlib. (We don't care what the function $g$ is, just that it's holomorphic.)
\end{proof}


\begin{theorem}[HolomorphicOn.vanishesOnRectangle]\label{HolomorphicOn.vanishesOnRectangle}\lean{HolomorphicOn.vanishesOnRectangle}\leanok
If $f$ is holomorphic on a rectangle $z$ and $w$, then the integral of $f$ over the rectangle with corners $z$ and $w$ is $0$.
\end{theorem}


\begin{proof}\leanok
This is in a Mathlib PR.
\end{proof}


The next lemma allows to zoom a big rectangle down to a small square, centered at a pole.

\begin{lemma}[RectanglePullToNhdOfPole]\label{RectanglePullToNhdOfPole}\lean{RectanglePullToNhdOfPole}\leanok
If $f$ is holomorphic on a rectangle $z$ and $w$ except at a point $p$, then the integral of $f$
over the rectangle with corners $z$ and $w$ is the same as the integral of $f$ over a small square
centered at $p$.
\end{lemma}


\begin{proof}\uses{HolomorphicOn.vanishesOnRectangle}\leanok
Chop the big rectangle with two vertical cuts and two horizontal cuts into smaller rectangles,
the middle one being the desired square. The integral over each of the outer rectangles
vanishes, since $f$ is holomorphic there. (The constant $c$ being ``small enough'' here just means
that the inner square is strictly contained in the big rectangle.)

\end{proof}


\begin{lemma}[ResidueTheoremAtOrigin]\label{ResidueTheoremAtOrigin}
\lean{ResidueTheoremAtOrigin}\leanok
The rectangle (square) integral of $f(s) = 1/s$ with corners $-1-i$ and $1+i$ is equal to $2\pi i$.
\end{lemma}


\begin{proof}\leanok
This is a special case of the more general result above.
\end{proof}


\begin{lemma}[ResidueTheoremOnRectangleWithSimplePole]\label{ResidueTheoremOnRectangleWithSimplePole}
\lean{ResidueTheoremOnRectangleWithSimplePole}\leanok
Suppose that $f$ is a holomorphic function on a rectangle, except for a simple pole
at $p$. By the latter, we mean that there is a function $g$ holomorphic on the rectangle such that, $f = g + A/(s-p)$ for some $A\in\C$. Then the integral of $f$ over the
rectangle is $A$.
\end{lemma}


\begin{proof}
\uses{ResidueTheoremAtOrigin, RectanglePullToNhdOfPole, HolomorphicOn.vanishesOnRectangle}
\leanok
Replace $f$ with $g + A/(s-p)$ in the integral.
The integral of $g$ vanishes by Lemma \ref{HolomorphicOn.vanishesOnRectangle}.
 To evaluate the integral of $1/(s-p)$,
pull everything to a square about the origin using Lemma \ref{RectanglePullToNhdOfPole},
and rescale by $c$;
what remains is handled by Lemma \ref{ResidueTheoremAtOrigin}.
\end{proof}

