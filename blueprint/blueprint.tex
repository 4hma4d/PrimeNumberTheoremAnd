\documentclass{report}

\usepackage{amsmath, amsthm}
\usepackage[showmore, dep_graph, coverage, project=../../]{blueprint}

\theoremstyle{definition}
\newtheorem{definition}{Definition}
\newtheorem{theorem}{Theorem}
\newtheorem{proposition}{Proposition}
\newtheorem{lemma}{Lemma}
\newtheorem{corollary}{Corollary}

\dochome{https://github.com/AlexKontorovich/PrimeNumberTheoremAnd/docs}

\title{Prime Number Theorem And ...}

\newcommand{\R}{\mathbb{R}}
\newcommand{\Q}{\mathbb{Q}}
\newcommand{\C}{\mathbb{C}}
\newcommand{\Z}{\mathbb{Z}}
\newcommand{\N}{\mathbb{N}}


\begin{document}
\maketitle

\chapter{The project}


The first main goal is to prove the Prime Number Theorem. Continuations of this project aim to extend
this work to primes in progressions (Dirichlet's theorem), Chebytarev's density theorem, etc
etc.

There are (at least) three approaches to PNT that we may want to pursue simultaneously. The quickest, at this stage, is likely to 
follow
 the ``Euclidean Products'' project by Michael Stoll, which has a proof of PNT missing only the Wiener-Ikehara Tauberian theorem. 

The second develops some complex analysis (residue calculus on rectangles, argument principle, Mellin transforms), to pull contours and derive a PNT with just an asymptotic formula.

The third approach, which will be the most general of the three, is to: (1) develop the residue calculus et al, as above, (2) add the Hadamard factorization theorem, (3) use it to prove the zero-free region for zeta via Hoffstein-Lockhart+Goldfeld-Hoffstein-Liemann (which generalizes to higher degree L-functions), and (4) use this to prove the prime number theorem with exp-root-log savings.

\chapter{First approach: Wiener-Ikehara Tauberian theorem.}


The Fourier transform of an absolutely integrable function $\psi: \R \to \C$ is defined by the formula
$$ \hat \psi(u) := \int_\R e(-tu) \psi(t)\ dt$$
where $e(\theta) := e^{2\pi i \theta}$.

Let $f: \N \to \C$ be an arithmetic function such that $\sum_{n=1}^\infty \frac{|f(n)|}{n^\sigma} < \infty$ for all $\sigma>1$.  Then the Dirichlet series
$$ F(s) := \sum_{n=1}^\infty \frac{f(n)}{n^s}$$
is absolutely convergent for $\sigma>1$.



\begin{lemma}[First Fourier identity]\label{first-fourier}  If $\psi: \R \to \C$ is continuous and compactly supported and $x > 0$, then for any $\sigma>1$
  $$ \sum_{n=1}^\infty \frac{f(n)}{n^\sigma} \hat \psi( \frac{1}{2\pi} \log \frac{n}{x} ) = \int_\R F(\sigma + it) \psi(t) x^{it}\ dt.$$
\end{lemma}



\begin{proof}  By the definition of the Fourier transform, the left-hand side expands as
$$ \sum_{n=1}^\infty \int_\R \frac{f(n)}{n^\sigma} \psi(t) e( - \frac{1}{2\pi} t \log \frac{n}{x})\ dt$$
while the right-hand side expands as
$$ \int_\R \sum_{n=1}^\infty \frac{f(n)}{n^{\sigma+it}} \psi(t) x^{it}\ dt.$$
Since
$$\frac{f(n)}{n^\sigma} \psi(t) e( - \frac{1}{2\pi} t \log \frac{n}{x}) = \frac{f(n)}{n^{\sigma+it}} \psi(t) x^{it}$$
the claim then follows from Fubini's theorem.
\end{proof}



\begin{lemma}[Second Fourier identity]\label{second-fourier} If $\psi: \R \to \C$ is continuous and compactly supported and $x > 0$, then for any $\sigma>1$
$$ \int_{-\log x}^\infty e^{-u(\sigma-1)} \hat \psi(\frac{u}{2\pi})\ du = x^{\sigma - 1} \int_\R \frac{1}{\sigma+it-1} \psi(t) x^{it}\ dt.$$
\end{lemma}



\begin{proof}
\uses{first-fourier}
  The left-hand side expands as
  $$ \int_{-\log x}^\infty \int_\R e^{-u(\sigma-1)} \psi(t) e(-\frac{tu}{2\pi})\ dt du = x^{\sigma - 1} \int_\R \frac{1}{\sigma+it-1} \psi(t) x^{it}\ dt$$
  so by Fubini's theorem it suffices to verify the identity
$$ \int_{-\log x}^\infty \int_\R e^{-u(\sigma-1)} e(-\frac{tu}{2\pi})\ du = x^{\sigma - 1} \frac{1}{\sigma+it-1} x^{it}$$
which is a routine calculation.
\end{proof}



Now let $A \in \C$, and suppose that there is a continuous function $G(s)$ defined on $\mathrm{Re} s \geq 1$ such that $G(s) = F(s) - \frac{A}{s-1}$ whenever $\mathrm{Re} s > 1$.  We also make the Chebyshev-type hypothesis
\begin{equation}\label{cheby}
\sum_{n \leq x} |f(n)| \ll x
\end{equation}
for all $x \geq 1$ (this hypothesis is not strictly necessary, but simplifies the arguments and can be obtained fairly easily in applications).



\begin{lemma}[Decay bounds]\label{decay}  If $\psi:\R \to \C$ is $C^2$ and obeys the bounds
  $$ |\psi(t)|, |\psi''(t)| \leq A / (1 + |t|^2)$$
  for all $t \in \R$, then
$$ |\hat \psi(u)| \leq C A / (1+|u|^2)$$
for all $u \in \R$, where $C$ is an absolute constant.
\end{lemma}



\begin{proof} This follows from a standard integration by parts argument.
\end{proof}



\begin{lemma}[Limiting Fourier identity]\label{limiting}  If $\psi: \R \to \C$ is $C^2$ and compactly supported and $x \geq 1$, then
$$ \sum_{n=1}^\infty \frac{f(n)}{n} \hat \psi( \frac{1}{2\pi} \log \frac{n}{x} ) - A \int_{-\log x}^\infty \hat \psi(\frac{u}{2\pi})\ du =  \int_\R G(1+it) \psi(t) x^{it}\ dt.$$
\end{lemma}



\begin{proof}
\uses{first-fourier,second-fourier,decay}
 By the preceding two lemmas, we know that for any $\sigma>1$, we have
  $$ \sum_{n=1}^\infty \frac{f(n)}{n^\sigma} \hat \psi( \frac{1}{2\pi} \log \frac{n}{x} ) - A x^{1-\sigma} \int_{-\log x}^\infty e^{-u(\sigma-1)} \hat \psi(\frac{u}{2\pi})\ du =  \int_\R G(\sigma+it) \psi(t) x^{it}\ dt.$$
  Now take limits as $\sigma \to 1$ using dominated convergence together with \eqref{cheby} and Lemma \ref{decay} to obtain the result.
\end{proof}



\begin{corollary}\label{limiting-cor}  With the hypotheses as above, we have
  $$ \sum_{n=1}^\infty \frac{f(n)}{n} \hat \psi( \frac{1}{2\pi} \log \frac{n}{x} ) = A \int_{-\infty}^\infty \hat \psi(\frac{u}{2\pi})\ du + o(1)$$
  as $x \to \infty$.
\end{corollary}



\begin{proof}
\uses{limiting}
 Immediate from the Riemann-Lebesgue lemma, and also noting that $\int_{-\infty}^{-\log x} \hat \psi(\frac{u}{2\pi})\ du = o(1)$.
\end{proof}



\begin{lemma}\label{schwarz-id}  The previous corollary also holds for functions $\psi$ that are assumed to be in the Schwartz class, as opposed to being $C^2$ and compactly supported.
\end{lemma}



\begin{proof}
\uses{limiting-cor}
For any $R>1$, one can use a smooth cutoff function to write $\psi = \psi_{\leq R} + \psi_{>R}$, where $\psi_{\leq R}$ is $C^2$ (in fact smooth) and compactly supported (on $[-R,R]$), and $\psi_{>R}$ obeys bounds of the form
$$ |\psi_{>R}(t)|, |\psi''_{>R}(t)| \ll R^{-1} / (1 + |t|^2) $$
where the implied constants depend on $\psi$.  By Lemma \ref{decay} we then have
$$ \hat \psi_{>R}(u) \ll R^{-1} / (1+|u|^2).$$
Using this and \eqref{cheby} one can show that
$$ \sum_{n=1}^\infty \frac{f(n)}{n} \hat \psi_{>R}( \frac{1}{2\pi} \log \frac{n}{x} ), A \int_{-\infty}^\infty \hat \psi_{>R} (\frac{u}{2\pi})\ du \ll R^{-1} $$
(with implied constants also depending on $A$), while from Lemma \ref{limiting-cor} one has
$$ \sum_{n=1}^\infty \frac{f(n)}{n} \hat \psi_{\leq R}( \frac{1}{2\pi} \log \frac{n}{x} ) = A \int_{-\infty}^\infty \hat \psi_{\leq R} (\frac{u}{2\pi})\ du + o(1).$$
Combining the two estimates and letting $R$ be large, we obtain the claim.
\end{proof}



\begin{lemma}\label{bij}  The Fourier transform is a bijection on the Schwartz class.
\end{lemma}



\begin{proof}  This is a standard result in Fourier analysis.
\end{proof}



\begin{corollary}\label{WienerIkeharaSmooth}
  If $\Psi: (0,\infty) \to \C$ is smooth and compactly supported away from the origin, then, then
$$ \sum_{n=1}^\infty f(n) \Psi( \frac{n}{x} ) = A x \int_0^\infty \Psi(y)\ dy + o(x)$$
as $u \to \infty$.
\end{corollary}



\begin{proof}
\uses{bij,schwarz-id}
 By Lemma \ref{bij}, we can write
$$ y \Psi(y) = \hat \psi( \frac{1}{2\pi} \log y )$$
for all $y>0$ and some Schwartz function $\psi$.  Making this substitution, the claim is then equivalent after standard manipulations to
$$ \sum_{n=1}^\infty \frac{f(n)}{n} \hat \psi( \frac{1}{2\pi} \log \frac{n}{x} ) = A \int_{-\infty}^\infty \hat \psi(\frac{u}{2\pi})\ du + o(1)$$
and the claim follows from Lemma \ref{schwarz-id}.
\end{proof}



\begin{lemma}[Smooth Urysohn lemma]\label{smooth-ury}  If $I$ is a closed interval contained in an open interval $J$, then there exists a smooth function $\Psi: \R \to \R$ with $1_I \leq \Psi \leq 1_J$.
\end{lemma}



\begin{proof}  A standard analysis lemma, which can be proven by convolving $1_K$ with a smooth approximation to the identity for some interval $K$ between $I$ and $J$.
\end{proof}



Now we add the hypothesis that $f(n) \geq 0$ for all $n$.

\begin{proposition}
\label{WienerIkeharaInterval}
  For any closed interval $I \subset (0,+\infty)$, we have
  $$ \sum_{n=1}^\infty f(n) 1_I( \frac{n}{x} ) = A x |I|  + o(x).$$
\end{proposition}



\begin{proof}
\uses{smooth-ury, WienerIkeharaSmooth}
  Use Lemma \ref{smooth-ury} to bound $1_I$ above and below by smooth compactly supported functions whose integral is close to the measure of $|I|$, and use the non-negativity of $f$.
\end{proof}



\begin{corollary}\label{WienerIkehara}
  We have
$$ \sum_{n\leq x} f(n) = A x |I|  + o(x).$$
\end{corollary}



\begin{proof}
\uses{WienerIkeharaInterval, cheby}
  Apply the preceding proposition with $I = [\varepsilon,1]$ and then send $\varepsilon$ to zero (using \eqref{cheby} to control the error).
\end{proof}



\section{Weak PNT}

\begin{theorem}[Weak PNT]\label{WeakPNT}  We have
$$ \sum_{n \leq x} \Lambda(n) = x + o(x).$$
\end{theorem}



\begin{proof}
\uses{WienerIkehara, ChebyshevPsi}
  Already done by Stoll, assuming Wiener-Ikehara.
\end{proof}




\chapter{Second approach}

\section{Residue calculus on rectangles}


In this file, we develop residue calculus on rectangles.

\begin{definition}\label{Rectangle}\lean{Rectangle}\leanok
A Rectangle has corners $z$ and $w \in \C$.
\end{definition}



The border of a rectangle is the union of its four sides.
\begin{definition}\label{RectangleBorder}\lean{RectangleBorder}\leanok
A Rectangle's border, given corners $z$ and $w$ is the union of the four sides.
\end{definition}



\begin{definition}\label{RectangleIntegral}\lean{RectangleIntegral}\leanok
A RectangleIntegral of a function $f$ is one over a rectangle determined by $z$ and $w$ in $\C$.
\end{definition}



\begin{definition}\label{MeromorphicOnRectangle}\lean{MeromorphicOnRectangle}\leanok
A function $f$ is Meromorphic on a rectangle with corners $z$ and $w$ if it is holomorphic off a
(finite) set of poles, none of which are on the boundary of the rectangle.
\end{definition}



\begin{theorem}\label{RectangleIntegralEqSumOfRectangles}\lean{RectangleIntegralEqSumOfRectangles}
If $f$ is meromorphic on a rectangle with corners $z$ and $w$, then the rectangle integral of $f$
is equal to the sum of sufficiently small rectangle integrals around each pole.
\end{theorem}



\begin{proof}
\uses{MeromorphicOnRectangle, RectangleIntegral}
Rectangles tile rectangles, zoom in.
\end{proof}



A meromorphic function has a pole of finite order.
\begin{definition}\label{PoleOrder}
If $f$ has a pole at $z_0$, then there is an integer $n$ such that
$$
\lim_{z\to z_0} (z-z_0)^n f(z) = c \neq 0.
$$
\end{definition}

[Note: There is a recent PR by David Loeffler dealing with orders of poles.]



If a meromorphic function $f$ has a pole at $z_0$, then the residue of $f$ at $z_0$ is the coefficient of $1/(z-z_0)$ in the Laurent series of $f$ around $z_0$.
\begin{definition}\label{Residue}
If $f$ has a pole of order $n$ at $z_0$, then
$$
Res_{z_0} f = \lim_{z\to z_0}\frac1{(n-1)!}(\partial/\partial z)^{n-1}[(z-z_0)^{n-1}f(z)].
$$
\end{definition}



We can evaluate a small integral around a pole by taking the residue.
\begin{theorem}\label{ResidueTheoremOnRectangle}\lean{ResidueTheoremOnRectangle}
If $f$ has a pole at $z_0$, then every small enough rectangle integral around $z_0$ is equal to $2\pi i Res_{z_0} f$.
\end{theorem}



\begin{proof}
\uses{PoleOrder, Residue, RectangleIntegral}
Near $z_0$, $f$ looks like $(z-z_0)^{-n} g(z)$, where $g$ is holomorphic and $g(z_0) \neq 0$.
Expand $g$ in a power series around $z_0$, so that
$$
f(z) = a_{-n}(z-z_0)^{-n} + \cdots + a_{-1}(z-z_0)^{-1} + \cdots.
$$
We can integrate term by term.

The key is being able to integrate $1/z$ around a rectangle with corners, say, $-1-i$ and $1+i$. The bottom is:
$$
\int_{-1-i}^{1-i} \frac1z dz
=
\int_{-1}^1 \frac1{x-i} dx,
$$
and the top is the negative of:
$$
\int_{-1+i}^{1+i} \frac1z dz
=
\int_{-1}^1 \frac1{x+i} dx.
$$
The two together add up to:
$$
\int_{-1}^1
\left(\frac1{x-i}-\frac1{x+i} \right)dx
=
2i\int_{-1}^1
\frac{1}{x^2+1}dx,
$$
which is the arctan at $1$ (namely $\pi/4$) minus that at $-1$. In total, this contributes $\pi i$ to the integral.

The vertical sides are:
$$
\int_{1-i}^{1+i} \frac1z dz
=
i\int_{-1}^1 \frac1{1+iy} dy
$$
and the negative of
$$
\int_{-1-i}^{-1+i} \frac1z dz
=
i\int_{-1}^1 \frac1{-1+iy} dy.
$$
This difference comes out to:
$$
i\int_{-1}^1 \left(\frac1{1+iy}-\frac1{-1+iy}\right) dy
=
i\int_{-1}^1 \left(\frac{-2}{-1-y^2}\right) dy,
$$
which contributes another factor of $\pi i$. (Fun! Each of the vertical/horizontal sides contributes half of the winding.)
\end{proof}

[Note: Of course the standard thing is to do this with circles, where the integral comes out directly from the parametrization. But discs don't tile
discs! Thus the standard approach is with annoying keyhole contours, etc; this is a total mess to formalize! Instead, we observe that rectangles do tile rectangles, so we can just do the
whole theory with rectangles. The cost is the extra difficulty of this little calculation.]

[Note: We only ever need simple poles for PNT, so would be enough to develop those...]



If a function $f$ is meromorphic at $z_0$ with a pole of order $n$, then
the residue at $z_0$ of the logarithmic derivative is $-n$ exactly.
\begin{theorem}\label{ResidueOfLogDerivative}\lean{ResidueOfLogDerivative}
If $f$ has a pole of order $n$ at $z_0$, then
$$
Res_{z_0} \frac{f'}f = -n.
$$
\end{theorem}



\begin{proof}
\uses{Residue, PoleOrder}
We can write $f(z) = (z-z_0)^{-n} g(z)$, where $g$ is holomorphic and $g(z_0) \neq 0$.
Then $f'(z) = -n(z-z_0)^{-n-1} g(z) + (z-z_0)^{-n} g'(z)$, so
$$
\frac{f'(z)}{f(z)} = \frac{-n}{z-z_0} + \frac{g'(z)}{g(z)}.
$$
The residue of the first term is $-n$, and the residue of the second term is $0$.
\end{proof}




\section{Mellin transforms}

In this section, we define the Mellin transform (already in Mathlib, thanks to David Loeffler), prove its inversion formula, and
derive a number of important properties of some special functions and bumpfunctions.

\begin{definition}\label{MellinTransform}
Let $f$ be a function from $\mathbb{R}_{>0}$ to $\mathbb{C}$. We define the Mellin transform of $f$ to be the function $\mathcal{M}(f)$ from $\mathbb{C}$ to $\mathbb{C}$ defined by
$$\mathcal{M}(f)(s) = \int_0^\infty f(x)x^{s-1}dx.$$
\end{definition}

[Note: My preferred way to think about this is that we are integrating over the multiplicative group $\mathbb{R}_{>0}$, multiplying by a (not necessarily unitary!) character $|\cdot|^s$, and integrating with respect to the invariant Haar measure $dx/x$. This is very useful in the kinds of calculations carried out below. But may be more difficult to formalize as things now stand. So we
might have clunkier calculations, which ``magically'' turn out just right - of course they're explained by the aforementioned structure...]




It is very convenient to define integrals along vertical lines in the complex plane, as follows.
\begin{definition}\label{VerticalIntegral}
Let $f$ be a function from $\mathbb{C}$ to $\mathbb{C}$, and let $\sigma$ be a real number. Then we define
$$\int_{(\sigma)}f(s)ds = \int_{\sigma-i\infty}^{\sigma+i\infty}f(s)ds.$$
\end{definition}
[Note: Better to define $\int_{(\sigma)}$ as $\frac1{2\pi i}\int_{\sigma-i\infty}^{\sigma+i\infty}$??
There's a factor of $2\pi i$ in such contour integrals...]



We first prove the following ``Perron-type'' formula.
\begin{lemma}\label{PerronFormula}
For $x>0$ and $\sigma>1$, we have
$$
\frac1{2\pi i}
\int_{(\sigma)}\frac{x^s}{s(s+1)}ds = \begin{cases}
1-\frac1x & \text{ if }x>1\\
0 & \text{ if } x<1
\end{cases}.
$$
\end{lemma}



\begin{proof}
\uses{ResidueTheoremOnRectangle, RectangleIntegralEqSumOfRectangles, VerticalIntegral, MellinTransform}
Pull contours and collect residues. This only involves rectangles, and everything is absolutely convergent.
\end{proof}



\begin{theorem}\label{MellinInversion}
Let $f$ be a nice function from $\mathbb{R}_{>0}$ to $\mathbb{C}$, and let $\sigma$ be sufficiently large. Then
$$f(x) = \frac{1}{2\pi i}\int_{(\sigma)}\mathcal{M}(f)(s)x^{-s}ds.$$
\end{theorem}

[Note: How ``nice''? Schwartz (on $(0,\infty)$) is certainly enough. As we formalize this, we can add whatever conditions are necessary for the proof to go through.]



\begin{proof}
\uses{PerronFormula}
The proof is from [Goldfeld-Kontorovich 2012].
Integrate by parts twice.
$$
\mathcal{M}(f)(s) = \int_0^\infty f(x)x^{s-1}dx = - \int_0^\infty f'(x)x^s\frac{1}{s}dx = \int_0^\infty f''(x)x^{s+1}\frac{1}{s(s+1)}dx.
$$
Assuming $f$ is Schwartz, say, we now have at least quadratic decay in $s$ of the Mellin transform. Inserting this formula into the inversion formula and Fubini-Tonelli (we now have absolute convergence!) gives:
$$
RHS = \frac{1}{2\pi i}\left(\int_{(\sigma)}\int_0^\infty f''(t)t^{s+1}\frac{1}{s(s+1)}dt\right) x^{-s}ds
$$
$$
= \int_0^\infty f''(t) t \left( \frac{1}{2\pi i}\int_{(\sigma)}(t/x)^s\frac{1}{s(s+1)}ds\right) dt.
$$
Apply the Perron formula to the inside:
$$
= \int_x^\infty f''(t) t \left(1-\frac{x}{t}\right)dt
= -\int_x^\infty f'(t) dt
= f(x),
$$
where we integrated by parts (undoing the first partial integration), and finally applied the fundamental theorem of calculus (undoing the second).
\end{proof}



Finally, we need Mellin Convolutions and properties thereof.
\begin{definition}\label{MellinConvolution}
Let $f$ and $g$ be functions from $\mathbb{R}_{>0}$ to $\mathbb{C}$. Then we define the Mellin convolution of $f$ and $g$ to be the function $f\ast g$ from $\mathbb{R}_{>0}$ to $\mathbb{C}$ defined by
$$(f\ast g)(x) = \int_0^\infty f(y)g(x/y)\frac{dy}{y}.$$
\end{definition}



The Mellin transform of a convolution is the product of the Mellin transforms.
\begin{theorem}\label{MellinConvolutionTransform}
Let $f$ and $g$ be functions from $\mathbb{R}_{>0}$ to $\mathbb{C}$. Then
$$\mathcal{M}(f\ast g)(s) = \mathcal{M}(f)(s)\mathcal{M}(g)(s).$$
\end{theorem}



\begin{proof}
\uses{MellinTransform}
This is a straightforward calculation.
\end{proof}



Let $\psi$ be a bumpfunction.
\begin{theorem}\label{SmoothExistence}
There exists a smooth (once differentiable would be enough), nonnegative ``bumpfunction'' $\psi$,
 supported in $[1/2,2]$ with total mass one:
$$
\int_0^\infty \psi(x)\frac{dx}{x} = 1.
$$
\end{theorem}
\begin{proof}
\uses{smooth-ury}
Same idea as Urysohn-type argument.
\end{proof}



The $\psi$ function has Mellin transform $\mathcal{M}(\psi)(s)$ which is entire and decays (at least) like $1/|s|$.
\begin{theorem}\label{MellinOfPsi}
The Mellin transform of $\psi$ is
$$\mathcal{M}(\psi)(s) =  O\left(\frac{1}{|s|}\right),$$
as $|s|\to\infty$.
\end{theorem}

[Of course it decays faster than any power of $|s|$, but it turns out that we will just need one power.]



\begin{proof}
\uses{MellinTransform, SmoothExistence}
Integrate by parts once.
\end{proof}



We can make a delta spike out of this bumpfunction, as follows.
\begin{definition}\label{DeltaSpike}
\uses{SmoothExistence}
Let $\psi$ be a bumpfunction supported in $[1/2,2]$. Then for any $\epsilon>0$, we define the delta spike $\psi_\epsilon$ to be the function from $\mathbb{R}_{>0}$ to $\mathbb{C}$ defined by
$$\psi_\epsilon(x) = \frac{1}{\epsilon}\psi\left(x^{\frac{1}{\epsilon}}\right).$$
\end{definition}

This spike still has mass one:
\begin{lemma}\label{DeltaSpikeMass}
For any $\epsilon>0$, we have
$$\int_0^\infty \psi_\epsilon(x)\frac{dx}{x} = 1.$$
\end{lemma}



\begin{proof}
\uses{DeltaSpike}
Substitute $y=x^{1/\epsilon}$, and use the fact that $\psi$ has mass one, and that $dx/x$ is Haar measure.
\end{proof}



The Mellin transform of the delta spike is easy to compute.
\begin{theorem}\label{MellinOfDeltaSpike}
For any $\epsilon>0$, the Mellin transform of $\psi_\epsilon$ is
$$\mathcal{M}(\psi_\epsilon)(s) = \mathcal{M}(\psi)\left(\epsilon s\right).$$
\end{theorem}



\begin{proof}
\uses{DeltaSpike, MellinTransform}
Substitute $y=x^{1/\epsilon}$, use Haar measure; direct calculation.
\end{proof}



In particular, for $s=1$, we have that the Mellin transform of $\psi_\epsilon$ is $1+O(\epsilon)$.
\begin{corollary}\label{MellinOfDeltaSpikeAt1}
For any $\epsilon>0$, we have
$$\mathcal{M}(\psi_\epsilon)(1) =
\mathcal{M}(\psi)(\epsilon)= 1+O(\epsilon).$$
\end{corollary}



\begin{proof}
\uses{MellinOfDeltaSpike, DeltaSpikeMass}
This is immediate from the above theorem, the fact that $\mathcal{M}(\psi)(0)=1$ (total mass one),
and that $\psi$ is Lipschitz.
\end{proof}



Let $1_{(0,1]}$ be the function from $\mathbb{R}_{>0}$ to $\mathbb{C}$ defined by
$$1_{(0,1]}(x) = \begin{cases}
1 & \text{ if }x\leq 1\\
0 & \text{ if }x>1
\end{cases}.$$
This has Mellin transform
\begin{theorem}\label{MellinOf1}
The Mellin transform of $1_{(0,1]}$ is
$$\mathcal{M}(1_{(0,1]})(s) = \frac{1}{s}.$$
\end{theorem}
[Note: this already exists in mathlib]



What will be essential for us is properties of the smooth version of $1_{(0,1]}$, obtained as the
 Mellin convolution of $1_{(0,1]}$ with $\psi_\epsilon$.
\begin{definition}\label{Smooth1}\uses{MellinOf1, MellinConvolution}
Let $\epsilon>0$. Then we define the smooth function $\widetilde{1_{\epsilon}}$ from $\mathbb{R}_{>0}$ to $\mathbb{C}$ by
$$\widetilde{1_{\epsilon}} = 1_{(0,1]}\ast\psi_\epsilon.$$
\end{definition}



In particular, we have the following
\begin{lemma}\label{Smooth1Properties}
Fix $\epsilon>0$. There is an absolute constant $c>0$ so that:

(1) If $x\leq (1-c\epsilon)$, then
$$\widetilde{1_{\epsilon}}(x) = 1.$$

And (2):
if $x\geq (1+c\epsilon)$, then
$$\widetilde{1_{\epsilon}}(x) = 0.$$
\end{lemma}



\begin{proof}
\uses{Smooth1, MellinConvolution}
This is a straightforward calculation, using the fact that $\psi_\epsilon$ is supported in $[1/2^\epsilon,2^\epsilon]$.
\end{proof}



Combining the above, we have the following Main Lemma of this section on the Mellin transform of $\widetilde{1_{\epsilon}}$.
\begin{lemma}\label{MellinOfSmooth1}\uses{Smooth1Properties, MellinConvolutionTransform, MellinOfDeltaSpikeAt1}
Fix  $\epsilon>0$. Then the Mellin transform of $\widetilde{1_{\epsilon}}$ is
$$\mathcal{M}(\widetilde{1_{\epsilon}})(s) = \frac{1}{s}\left(\mathcal{M}(\psi)\left(\epsilon s\right)\right).$$

For any $s$, we have the bound
$$\mathcal{M}(\widetilde{1_{\epsilon}})(s) = O\left(\frac{1}{\epsilon|s|^2}\right).$$

At $s=1$, we have
$$\mathcal{M}(\widetilde{1_{\epsilon}})(1) = (1+O(\epsilon)).$$
\end{lemma}




\section{Second Proof of PNT}

The approach here is completely standard. We follow the use of $\mathcal{M}(\widetilde{1_{\epsilon}})$ as in Kontorovich 2015.




It has already been established that zeta doesn't vanish on the 1 line, and has a pole at $s=1$ of order 1.
We also have that
$$
-\frac{\zeta'(s)}{\zeta(s)} = \sum_{n=1}^\infty \frac{\Lambda(n)}{n^s}.
$$

The main object of study is the following inverse Mellin-type transform, which will turn out to be a smoothed Chebyshev function.
\begin{definition}\label{SmoothedChebyshev}
Fix $\epsilon>0$, and a bumpfunction $\psi$ supported in $[1/2,2]$. Then we define the smoothed Chebyshev function $\psi_{\epsilon}$ from $\mathbb{R}_{>0}$ to $\mathbb{C}$ by
$$\psi_{\epsilon}(X) = \frac{1}{2\pi i}\int_{(2)}\frac{-\zeta'(s)}{\zeta(s)}
\mathcal{M}(\widetilde{1_{\epsilon}})(s)
X^{s}ds.$$
\end{definition}



Inserting the Dirichlet series expansion of the log derivative of zeta, we get the following.
\begin{theorem}\label{SmoothedChebyshevDirichlet}
We have that
$$\psi_{\epsilon}(X) = \sum_{n=1}^\infty \Lambda(n)\widetilde{1_{\epsilon}}(n/X).$$
\end{theorem}



\begin{proof}
We have that
$$\psi_{\epsilon}(X) = \frac{1}{2\pi i}\int_{(2)}\sum_{n=1}^\infty \frac{\Lambda(n)}{n^s}
\mathcal{M}(\widetilde{1_{\epsilon}})(s)
X^{s}ds.$$
We have enough decay (thanks to quadratic decay of $\mathcal{M}(\widetilde{1_{\epsilon}})$) to justify the interchange of summation and integration. We then get
$$\psi_{\epsilon}(X) =
\sum_{n=1}^\infty \Lambda(n)\frac{1}{2\pi i}\int_{(2)}
\mathcal{M}(\widetilde{1_{\epsilon}})(s)
(n/X)^{-s}
ds
$$
and apply the Mellin inversion formula (Theorem \ref{MellinInversion}).
\end{proof}



The smoothed Chebyshev function is close to the actual Chebyshev function.
\begin{theorem}\label{SmoothedChebyshevClose}
We have that
$$\psi_{\epsilon}(X) = \psi(X) + O(\epsilon X \log X).$$
\end{theorem}



\begin{proof}
Take the difference. By Lemma \ref{Smooth1Properties}, the sums agree except when $1-c \epsilon \leq n/X \leq 1+c \epsilon$. This is an interval of length $\ll \epsilon X$, and the summands are bounded by $\Lambda(n) \ll \log X$.
\end{proof}



Returning to the definition of $\psi_{\epsilon}$, fix a large $T$ to be chosen later, and pull contours (via rectangles!) to go
from $2$ up to $2+iT$, then over to $1+iT$, and up from there to $1+i\infty$ (and symmetrically in the lower half plane). Call
this path $\gamma$. The
rectangles involved are all where the integrand is holomorphic, so there is no change.
\begin{theorem}\label{SmoothedChebyshevPull1}
We have that
$$\psi_{\epsilon}(X) = \frac{1}{2\pi i}\int_{\gamma}\frac{-\zeta'(s)}{\zeta(s)}
\mathcal{M}(\widetilde{1_{\epsilon}})(s)
X^{s}ds.$$
\end{theorem}



Then, since $\zeta$ doesn't vanish on the 1-line, there is a $\delta$ (depending on $T$), so that the box $[1-\delta,1] \times_{ℂ} [-T,T]$ is free of zeros of $\zeta$.

The rectangle integral with opposite corners $1-\delta - i T$ and $2+iT$ contains a single pole of $-\zeta'/\zeta$ at $s=1$, and the residue is $1$ (from Theorem \ref{ResidueOfLogDerivative}).





\chapter{Third Approach}

\section{Hadamard factorization}


In this file, we prove the Hadamard Factorization theorem for functions of finite order, and prove that the zeta function
is such.




\section{Hoffstein-Lockhart}


In this file, we use the Hoffstein-Lockhart construction to prove a zero-free region for zeta.

ZeroFreeRegion

Hoffstein-Lockhart + Goldfeld-Hoffstein-Liemann





\section{Strong PNT}

\begin{definition}\label{ChebyshevPsi}\lean{ChebyshevPsi}\leanok
The Chebyshev Psi function is defined as
$$
\psi(x) = \sum_{n \leq x} \Lambda(n),
$$
where $\Lambda(n)$ is the von Mangoldt function.
\end{definition}




Main Theorem: The Prime Number Theorem in strong form.
\begin{theorem}[PrimeNumberTheorem]\label{StrongPNT}\lean{PrimeNumberTheorem}
There is a constant $c > 0$ such that
$$
ψ (x) = x + O(x e^{-c \sqrt{\log x}})
$$
as $x\to \infty$.
\end{theorem}




\end{document}
