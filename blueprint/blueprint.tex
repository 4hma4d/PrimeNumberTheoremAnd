\usepackage{amsmath, amsthm}

\theoremstyle{definition}
\newtheorem{definition}{Definition}
\newtheorem{theorem}{Theorem}
\newtheorem{proposition}{Proposition}
\newtheorem{lemma}{Lemma}
\newtheorem{corollary}{Corollary}

\title{Prime Number Theorem And ...}

\newcommand{\eps}{\epsilon}

\newcommand{\R}{\mathbb{R}}
\newcommand{\Q}{\mathbb{Q}}
\newcommand{\C}{\mathbb{C}}
\newcommand{\Z}{\mathbb{Z}}
\newcommand{\N}{\mathbb{N}}


\begin{document}
\maketitle

\chapter{The project}

The project github page is https://github.com/AlexKontorovich/PrimeNumberTheoremAnd.

The project docs page is https://alexkontorovich.github.io/PrimeNumberTheoremAnd/docs.

The first main goal is to prove the Prime Number Theorem in Lean.
(This remains one of the outstanding problems on Wiedijk's list of 100 theorems to formalize.)
Note that PNT has been formalized before, first by Avigad et al in Isabelle,
https://arxiv.org/abs/cs/0509025
following the Selberg / Erdos method, then
by Harrison in HOL Light
https://www.cl.cam.ac.uk/$\sim$jrh13/papers/mikefest.html
via Newman's proof.
Carniero gave another formalization in Metamath of the Selberg / Erdos method:
https://arxiv.org/abs/1608.02029,
and Eberl-Paulson gave a formalization of Newman's proof in Isabelle:
https://www.isa-afp.org/entries/Prime\_Number\_Theorem.html

Continuations of this project aim to extend
this work to primes in progressions (Dirichlet's theorem), Chebotarev's density theorem, etc
etc.

There are (at least) three approaches to PNT that we may want to pursue simultaneously. The quickest, at this stage, is likely to
follow
 the ``Euler Products'' project by Michael Stoll, which has a proof of PNT missing only the Wiener-Ikehara Tauberian theorem.

The second develops some complex analysis (residue calculus on rectangles, argument principle, Mellin transforms), to pull contours and derive a PNT with an error term which is stronger than any power of log savings.

The third approach, which will be the most general of the three, is to: (1) develop the residue calculus et al, as above, (2) add the Hadamard factorization theorem, (3) use it to prove the zero-free region for zeta via Hoffstein-Lockhart+Goldfeld-Hoffstein-Liemann (which generalizes to higher degree L-functions), and (4) use this to prove the prime number theorem with exp-root-log savings.

A word about the expected ``rate-limiting-steps'' in each of the approaches.

(*) In approach (1), I think it will be the fact that the Fourier transform is a bijection on the Schwartz class. There is a recent PR (https://github.com/leanprover-community/mathlib4/pull/9773) with David Loeffler and Heather Macbeth making the first steps in that direction, just computing the (Frechet) derivative of the Fourier transform. One will need to iterate on that to get arbitrary derivatives, to conclude that the transform of a Schwartz function is Schwartz. Then to get the bijection, we need an inversion formula. We can derive Fourier inversion *from* Mellin inversion! So it seems that the most important thing to start is Perron's formula.

(*) In approach (2), there are two rate-limiting-steps, neither too serious (in my estimation). The first is how to handle meromorphic functions on rectangles. Perhaps in this project, it should not be done in any generality, but on a case by case basis. There are two simple poles whose residues need to be computed in the proof of the Perron formula, and one simple pole in the log-derivative of zeta, nothing too complicated, and maybe we shouldn't get bogged down in the general case. The other is the fact that the $\epsilon$-smoothed Chebyshev function differs from the unsmoothed by $\epsilon X$ (and not $\epsilon X \log X$, as follows from a trivial bound). This needs a Brun-Titchmarsh type theorem, perhaps can be done even more easily in this case with a Selberg sieve, on which there is (partial?) progress in Mathlib.

(*) In approach (3), it's obviously the Hadamard factorization, which needs quite a lot of nontrivial mathematics. (But after that, the math is not hard, on top of things in approach (2) -- and if we're getting the strong error term, we can afford to lose $\log X$ in the Chebyshev discussion above...).

\chapter{First approach: Wiener-Ikehara Tauberian theorem.}

\section{A Fourier-analytic proof of the Wiener-Ikehara theorem}

The Fourier transform of an absolutely integrable function $\psi: \R \to \C$ is defined by the formula
$$ \hat \psi(u) := \int_\R e(-tu) \psi(t)\ dt$$
where $e(\theta) := e^{2\pi i \theta}$.

Let $f: \N \to \C$ be an arithmetic function such that $\sum_{n=1}^\infty \frac{|f(n)|}{n^\sigma} < \infty$ for all $\sigma>1$.  Then the Dirichlet series
$$ F(s) := \sum_{n=1}^\infty \frac{f(n)}{n^s}$$
is absolutely convergent for $\sigma>1$.

\begin{lemma}[First Fourier identity]\label{first-fourier}  If $\psi: \R \to \C$ is continuous and compactly supported and $x > 0$, then for any $\sigma>1$
  $$ \sum_{n=1}^\infty \frac{f(n)}{n^\sigma} \hat \psi( \frac{1}{2\pi} \log \frac{n}{x} ) = \int_\R F(\sigma + it) \psi(t) x^{it}\ dt.$$
\end{lemma}

\begin{proof}  By the definition of the Fourier transform, the left-hand side expands as
$$ \sum_{n=1}^\infty \int_\R \frac{f(n)}{n^\sigma} \psi(t) e( - \frac{1}{2\pi} t \log \frac{n}{x})\ dt$$
while the right-hand side expands as
$$ \int_\R \sum_{n=1}^\infty \frac{f(n)}{n^{\sigma+it}} \psi(t) x^{it}\ dt.$$
Since
$$\frac{f(n)}{n^\sigma} \psi(t) e( - \frac{1}{2\pi} t \log \frac{n}{x}) = \frac{f(n)}{n^{\sigma+it}} \psi(t) x^{it}$$
the claim then follows from Fubini's theorem.
\end{proof}

\begin{lemma}[Second Fourier identity]\label{second-fourier} If $\psi: \R \to \C$ is continuous and compactly supported and $x > 0$, then for any $\sigma>1$
$$ \int_{-\log x}^\infty e^{-u(\sigma-1)} \hat \psi(\frac{u}{2\pi})\ du = x^{\sigma - 1} \int_\R \frac{1}{\sigma+it-1} \psi(t) x^{it}\ dt.$$
\end{lemma}

\begin{proof}
\uses{first-fourier}
  The left-hand side expands as
  $$ \int_{-\log x}^\infty \int_\R e^{-u(\sigma-1)} \psi(t) e(-\frac{tu}{2\pi})\ dt du = x^{\sigma - 1} \int_\R \frac{1}{\sigma+it-1} \psi(t) x^{it}\ dt$$
  so by Fubini's theorem it suffices to verify the identity
$$ \int_{-\log x}^\infty \int_\R e^{-u(\sigma-1)} e(-\frac{tu}{2\pi})\ du = x^{\sigma - 1} \frac{1}{\sigma+it-1} x^{it}$$
which is a routine calculation.
\end{proof}

Now let $A \in \C$, and suppose that there is a continuous function $G(s)$ defined on $\mathrm{Re} s \geq 1$ such that $G(s) = F(s) - \frac{A}{s-1}$ whenever $\mathrm{Re} s > 1$.  We also make the Chebyshev-type hypothesis
\begin{equation}\label{cheby}
\sum_{n \leq x} |f(n)| \ll x
\end{equation}
for all $x \geq 1$ (this hypothesis is not strictly necessary, but simplifies the arguments and can be obtained fairly easily in applications).

\begin{lemma}[Decay bounds]\label{decay}  If $\psi:\R \to \C$ is $C^2$ and obeys the bounds
  $$ |\psi(t)|, |\psi''(t)| \leq A / (1 + |t|^2)$$
  for all $t \in \R$, then
$$ |\hat \psi(u)| \leq C A / (1+|u|^2)$$
for all $u \in \R$, where $C$ is an absolute constant.
\end{lemma}

\begin{proof} This follows from a standard integration by parts argument.
\end{proof}

\begin{lemma}[Limiting Fourier identity]\label{limiting}  If $\psi: \R \to \C$ is $C^2$ and compactly supported and $x \geq 1$, then
$$ \sum_{n=1}^\infty \frac{f(n)}{n} \hat \psi( \frac{1}{2\pi} \log \frac{n}{x} ) - A \int_{-\log x}^\infty \hat \psi(\frac{u}{2\pi})\ du =  \int_\R G(1+it) \psi(t) x^{it}\ dt.$$
\end{lemma}

\begin{proof}
\uses{first-fourier,second-fourier,decay}
 By the preceding two lemmas, we know that for any $\sigma>1$, we have
  $$ \sum_{n=1}^\infty \frac{f(n)}{n^\sigma} \hat \psi( \frac{1}{2\pi} \log \frac{n}{x} ) - A x^{1-\sigma} \int_{-\log x}^\infty e^{-u(\sigma-1)} \hat \psi(\frac{u}{2\pi})\ du =  \int_\R G(\sigma+it) \psi(t) x^{it}\ dt.$$
  Now take limits as $\sigma \to 1$ using dominated convergence together with \eqref{cheby} and Lemma \ref{decay} to obtain the result.
\end{proof}

\begin{corollary}\label{limiting-cor}  With the hypotheses as above, we have
  $$ \sum_{n=1}^\infty \frac{f(n)}{n} \hat \psi( \frac{1}{2\pi} \log \frac{n}{x} ) = A \int_{-\infty}^\infty \hat \psi(\frac{u}{2\pi})\ du + o(1)$$
  as $x \to \infty$.
\end{corollary}

\begin{proof}
\uses{limiting}
 Immediate from the Riemann-Lebesgue lemma, and also noting that $\int_{-\infty}^{-\log x} \hat \psi(\frac{u}{2\pi})\ du = o(1)$.
\end{proof}

\begin{lemma}\label{schwarz-id}  The previous corollary also holds for functions $\psi$ that are assumed to be in the Schwartz class, as opposed to being $C^2$ and compactly supported.
\end{lemma}

\begin{proof}
\uses{limiting-cor}
For any $R>1$, one can use a smooth cutoff function to write $\psi = \psi_{\leq R} + \psi_{>R}$, where $\psi_{\leq R}$ is $C^2$ (in fact smooth) and compactly supported (on $[-R,R]$), and $\psi_{>R}$ obeys bounds of the form
$$ |\psi_{>R}(t)|, |\psi''_{>R}(t)| \ll R^{-1} / (1 + |t|^2) $$
where the implied constants depend on $\psi$.  By Lemma \ref{decay} we then have
$$ \hat \psi_{>R}(u) \ll R^{-1} / (1+|u|^2).$$
Using this and \eqref{cheby} one can show that
$$ \sum_{n=1}^\infty \frac{f(n)}{n} \hat \psi_{>R}( \frac{1}{2\pi} \log \frac{n}{x} ), A \int_{-\infty}^\infty \hat \psi_{>R} (\frac{u}{2\pi})\ du \ll R^{-1} $$
(with implied constants also depending on $A$), while from Lemma \ref{limiting-cor} one has
$$ \sum_{n=1}^\infty \frac{f(n)}{n} \hat \psi_{\leq R}( \frac{1}{2\pi} \log \frac{n}{x} ) = A \int_{-\infty}^\infty \hat \psi_{\leq R} (\frac{u}{2\pi})\ du + o(1).$$
Combining the two estimates and letting $R$ be large, we obtain the claim.
\end{proof}

\begin{lemma}\label{bij}  The Fourier transform is a bijection on the Schwartz class.
\end{lemma}

\begin{proof}  This is a standard result in Fourier analysis.
\end{proof}

\begin{corollary}\label{WienerIkeharaSmooth}
  If $\Psi: (0,\infty) \to \C$ is smooth and compactly supported away from the origin, then, then
$$ \sum_{n=1}^\infty f(n) \Psi( \frac{n}{x} ) = A x \int_0^\infty \Psi(y)\ dy + o(x)$$
as $u \to \infty$.
\end{corollary}

\begin{proof}
\uses{bij,schwarz-id}
 By Lemma \ref{bij}, we can write
$$ y \Psi(y) = \hat \psi( \frac{1}{2\pi} \log y )$$
for all $y>0$ and some Schwartz function $\psi$.  Making this substitution, the claim is then equivalent after standard manipulations to
$$ \sum_{n=1}^\infty \frac{f(n)}{n} \hat \psi( \frac{1}{2\pi} \log \frac{n}{x} ) = A \int_{-\infty}^\infty \hat \psi(\frac{u}{2\pi})\ du + o(1)$$
and the claim follows from Lemma \ref{schwarz-id}.
\end{proof}

\begin{lemma}[Smooth Urysohn lemma]\label{smooth-ury}  If $I$ is a closed interval contained in an open interval $J$, then there exists a smooth function $\Psi: \R \to \R$ with $1_I \leq \Psi \leq 1_J$.
\end{lemma}

\begin{proof}  A standard analysis lemma, which can be proven by convolving $1_K$ with a smooth approximation to the identity for some interval $K$ between $I$ and $J$.
\end{proof}

Now we add the hypothesis that $f(n) \geq 0$ for all $n$.

\begin{proposition}
\label{WienerIkeharaInterval}
  For any closed interval $I \subset (0,+\infty)$, we have
  $$ \sum_{n=1}^\infty f(n) 1_I( \frac{n}{x} ) = A x |I|  + o(x).$$
\end{proposition}

\begin{proof}
\uses{smooth-ury}
  Use Lemma \ref{smooth-ury} to bound $1_I$ above and below by smooth compactly supported functions whose integral is close to the measure of $|I|$, and use the non-negativity of $f$.
\end{proof}

\begin{corollary}\label{WienerIkehara}
  We have
$$ \sum_{n\leq x} f(n) = A x |I|  + o(x).$$
\end{corollary}



\begin{proof}
\uses{WienerIkeharaInterval, cheby}
  Apply the preceding proposition with $I = [\varepsilon,1]$ and then send $\varepsilon$ to zero (using \eqref{cheby} to control the error).
\end{proof}





\chapter{Second approach}

\section{Residue calculus on rectangles}

This files gathers definitions and basic properties about rectangles.


\begin{definition}\label{Rectangle}\lean{Rectangle}\leanok
A Rectangle has corners $z$ and $w \in \C$.
\end{definition}


The border of a rectangle is the union of its four sides.
\begin{definition}[RectangleBorder]\label{RectangleBorder}\lean{RectangleBorder}\leanok
A Rectangle's border, given corners $z$ and $w$ is the union of the four sides.
\end{definition}




In this file, we develop residue calculus on rectangles.

\begin{definition}\label{Rectangle}\lean{Rectangle}\leanok
A Rectangle has corners $z$ and $w \in \C$.
\end{definition}



The border of a rectangle is the union of its four sides.
\begin{definition}\label{RectangleBorder}\lean{RectangleBorder}\leanok
A Rectangle's border, given corners $z$ and $w$ is the union of the four sides.
\end{definition}



\begin{definition}\label{RectangleIntegral}\lean{RectangleIntegral}\leanok
A RectangleIntegral of a function $f$ is one over a rectangle determined by $z$ and $w$ in $\C$.
\end{definition}



\begin{definition}\label{MeromorphicOnRectangle}\lean{MeromorphicOnRectangle}\leanok
A function $f$ is Meromorphic on a rectangle with corners $z$ and $w$ if it is holomorphic off a
(finite) set of poles, none of which are on the boundary of the rectangle.
\end{definition}



\begin{theorem}\label{RectangleIntegralEqSumOfRectangles}\lean{RectangleIntegralEqSumOfRectangles}
If $f$ is meromorphic on a rectangle with corners $z$ and $w$, then the rectangle integral of $f$
is equal to the sum of sufficiently small rectangle integrals around each pole.
\end{theorem}



\begin{proof}
\uses{MeromorphicOnRectangle, RectangleIntegral}
Rectangles tile rectangles, zoom in.
\end{proof}



A meromorphic function has a pole of finite order.
\begin{definition}\label{PoleOrder}
If $f$ has a pole at $z_0$, then there is an integer $n$ such that
$$
\lim_{z\to z_0} (z-z_0)^n f(z) = c \neq 0.
$$
\end{definition}

[Note: There is a recent PR by David Loeffler dealing with orders of poles.]



If a meromorphic function $f$ has a pole at $z_0$, then the residue of $f$ at $z_0$ is the coefficient of $1/(z-z_0)$ in the Laurent series of $f$ around $z_0$.
\begin{definition}\label{Residue}
If $f$ has a pole of order $n$ at $z_0$, then
$$
Res_{z_0} f = \lim_{z\to z_0}\frac1{(n-1)!}(\partial/\partial z)^{n-1}[(z-z_0)^{n-1}f(z)].
$$
\end{definition}



We can evaluate a small integral around a pole by taking the residue.
\begin{theorem}\label{ResidueTheoremOnRectangle}\lean{ResidueTheoremOnRectangle}
If $f$ has a pole at $z_0$, then every small enough rectangle integral around $z_0$ is equal to $2\pi i Res_{z_0} f$.
\end{theorem}



\begin{proof}
\uses{PoleOrder, Residue, RectangleIntegral}
Near $z_0$, $f$ looks like $(z-z_0)^{-n} g(z)$, where $g$ is holomorphic and $g(z_0) \neq 0$.
Expand $g$ in a power series around $z_0$, so that
$$
f(z) = a_{-n}(z-z_0)^{-n} + \cdots + a_{-1}(z-z_0)^{-1} + \cdots.
$$
We can integrate term by term.

The key is being able to integrate $1/z$ around a rectangle with corners, say, $-1-i$ and $1+i$. The bottom is:
$$
\int_{-1-i}^{1-i} \frac1z dz
=
\int_{-1}^1 \frac1{x-i} dx,
$$
and the top is the negative of:
$$
\int_{-1+i}^{1+i} \frac1z dz
=
\int_{-1}^1 \frac1{x+i} dx.
$$
The two together add up to:
$$
\int_{-1}^1
\left(\frac1{x-i}-\frac1{x+i} \right)dx
=
2i\int_{-1}^1
\frac{1}{x^2+1}dx,
$$
which is the arctan at $1$ (namely $\pi/4$) minus that at $-1$. In total, this contributes $\pi i$ to the integral.

The vertical sides are:
$$
\int_{1-i}^{1+i} \frac1z dz
=
i\int_{-1}^1 \frac1{1+iy} dy
$$
and the negative of
$$
\int_{-1-i}^{-1+i} \frac1z dz
=
i\int_{-1}^1 \frac1{-1+iy} dy.
$$
This difference comes out to:
$$
i\int_{-1}^1 \left(\frac1{1+iy}-\frac1{-1+iy}\right) dy
=
i\int_{-1}^1 \left(\frac{-2}{-1-y^2}\right) dy,
$$
which contributes another factor of $\pi i$. (Fun! Each of the vertical/horizontal sides contributes half of the winding.)
\end{proof}

[Note: Of course the standard thing is to do this with circles, where the integral comes out directly from the parametrization. But discs don't tile
discs! Thus the standard approach is with annoying keyhole contours, etc; this is a total mess to formalize! Instead, we observe that rectangles do tile rectangles, so we can just do the
whole theory with rectangles. The cost is the extra difficulty of this little calculation.]

[Note: We only ever need simple poles for PNT, so would be enough to develop those...]



If a function $f$ is meromorphic at $z_0$ with a pole of order $n$, then
the residue at $z_0$ of the logarithmic derivative is $-n$ exactly.
\begin{theorem}\label{ResidueOfLogDerivative}\lean{ResidueOfLogDerivative}
If $f$ has a pole of order $n$ at $z_0$, then
$$
Res_{z_0} \frac{f'}f = -n.
$$
\end{theorem}



\begin{proof}
\uses{Residue, PoleOrder}
We can write $f(z) = (z-z_0)^{-n} g(z)$, where $g$ is holomorphic and $g(z_0) \neq 0$.
Then $f'(z) = -n(z-z_0)^{-n-1} g(z) + (z-z_0)^{-n} g'(z)$, so
$$
\frac{f'(z)}{f(z)} = \frac{-n}{z-z_0} + \frac{g'(z)}{g(z)}.
$$
The residue of the first term is $-n$, and the residue of the second term is $0$.
\end{proof}




\section{Perron Formula}

In this section, we prove the Perron formula, which plays a key role in our proof of Mellin inversion.


The following is preparatory material used in the proof of the Perron formula, see Lemma \ref{formulaLtOne}.

\end{proof}

\end{proof}

\end{proof}


TODO : Move to general section
\begin{lemma}[limitOfConstant]\label{limitOfConstant}\lean{limitOfConstant}\leanok
Let $a:\R\to\C$ be a function, and let $\sigma>0$ be a real number. Suppose that, for all
$\sigma, \sigma'>0$, we have $a(\sigma')=a(\sigma)$, and that
$\lim_{\sigma\to\infty}a(\sigma)=0$. Then $a(\sigma)=0$.
\end{lemma}


\begin{proof}\leanok\begin{align*}
\lim_{\sigma'\to\infty}a(\sigma) &= \lim_{\sigma'\to\infty}a(\sigma') \\

 &= 0

\end{align*}\end{proof}


\begin{lemma}[limitOfConstantLeft]\label{limitOfConstantLeft}\lean{limitOfConstantLeft}\leanok
Let $a:\R\to\C$ be a function, and let $\sigma<-3/2$ be a real number. Suppose that, for all
$\sigma, \sigma'>0$, we have $a(\sigma')=a(\sigma)$, and that
$\lim_{\sigma\to-\infty}a(\sigma)=0$. Then $a(\sigma)=0$.
\end{lemma}


\begin{proof}\leanok
\begin{align*}
\lim_{\sigma'\to-\infty}a(\sigma) &= \lim_{\sigma'\to-\infty}a(\sigma') \\

 &= 0

\end{align*}\end{proof}


\begin{lemma}[tendsto_rpow_atTop_nhds_zero_of_norm_lt_one]\label{tendsto_rpow_atTop_nhds_zero_of_norm_lt_one}\lean{tendsto_rpow_atTop_nhds_zero_of_norm_lt_one}\leanok
Let $x>0$ and $x<1$. Then
$$\lim_{\sigma\to\infty}x^\sigma=0.$$
\end{lemma}


\begin{proof}\leanok
Standard.

\end{proof}


\begin{lemma}[tendsto_rpow_atTop_nhds_zero_of_norm_gt_one]\label{tendsto_rpow_atTop_nhds_zero_of_norm_gt_one}\lean{tendsto_rpow_atTop_nhds_zero_of_norm_gt_one}\leanok
Let $x>1$. Then
$$\lim_{\sigma\to-\infty}x^\sigma=0.$$
\end{lemma}


\begin{proof}\leanok
Standard.
\end{proof}


\begin{lemma}[isHolomorphicOn]\label{isHolomorphicOn}\lean{Perron.isHolomorphicOn}\leanok
Let $x>0$. Then the function $f(s) = x^s/(s(s+1))$ is holomorphic on the half-plane $\{s\in\mathbb{C}:\Re(s)>0\}$.
\end{lemma}


\begin{proof}\leanok
Composition of differentiabilities.

\end{proof}


\begin{lemma}[integralPosAux]\label{integralPosAux}\lean{Perron.integralPosAux}\leanok
The integral
$$\int_\R\frac{1}{|(1+t^2)(2+t^2)|^{1/2}}dt$$
is positive (and hence convergent - since a divergent integral is zero in Lean, by definition).
\end{lemma}


\begin{proof}\leanok
This integral is between $\frac{1}{2}$ and $1$ of the integral of $\frac{1}{1+t^2}$, which is $\pi$.

\end{proof}


\begin{lemma}[vertIntBound]\label{vertIntBound}\lean{Perron.vertIntBound}\leanok
Let $x>0$ and $\sigma>1$. Then
$$\left|
\int_{(\sigma)}\frac{x^s}{s(s+1)}ds\right| \leq x^\sigma \int_\R\frac{1}{|(1+t^2)(2+t^2)|^{1/2}}dt.$$
\end{lemma}


\begin{proof}\leanok
\uses{VerticalIntegral}
Triangle inequality and pointwise estimate.
\end{proof}


\begin{lemma}[vertIntBoundLeft]\label{vertIntBoundLeft}\lean{Perron.vertIntBoundLeft}\leanok
Let $x>1$ and $\sigma<-3/2$. Then
$$\left|
\int_{(\sigma)}\frac{x^s}{s(s+1)}ds\right| \leq x^\sigma \int_\R\frac{1}{|(1/4+t^2)(2+t^2)|^{1/2}}dt.$$
\end{lemma}


\begin{proof}\leanok
\uses{VerticalIntegral}


Triangle inequality and pointwise estimate.
\end{proof}


\begin{lemma}[isIntegrable]\label{isIntegrable}\lean{Perron.isIntegrable}\leanok
Let $x>0$ and $\sigma\in\R$. Then
$$\int_{\R}\frac{x^{\sigma+it}}{(\sigma+it)(1+\sigma + it)}d\sigma$$
is integrable.
\end{lemma}


\begin{proof}\uses{isHolomorphicOn}\leanok
By \ref{isHolomorphicOn}, $f$ is continuous, so it is integrable on any interval.

 Also, $|f(x)| = \Theta(x^{-2})$ as $x\to\infty$,

 and $|f(-x)| = \Theta(x^{-2})$ as $x\to\infty$.

 Since $g(x) = x^{-2}$ is integrable on $[a,\infty)$ for any $a>0$, we conclude.

\end{proof}


\begin{lemma}[tendsto_zero_Lower]\label{tendsto_zero_Lower}\lean{Perron.tendsto_zero_Lower}\leanok
Let $x>0$ and $\sigma',\sigma''\in\R$. Then
$$\int_{\sigma'}^{\sigma''}\frac{x^{\sigma+it}}{(\sigma+it)(1+\sigma + it)}d\sigma$$
goes to $0$ as $t\to-\infty$.
\end{lemma}


\begin{proof}\leanok
The numerator is bounded and the denominator tends to infinity.
\end{proof}


\begin{lemma}[tendsto_zero_Upper]\label{tendsto_zero_Upper}\lean{Perron.tendsto_zero_Upper}\leanok
Let $x>0$ and $\sigma',\sigma''\in\R$. Then
$$\int_{\sigma'}^{\sigma''}\frac{x^{\sigma+it}}{(\sigma+it)(1+\sigma + it)}d\sigma$$
goes to $0$ as $t\to\infty$.
\end{lemma}


\begin{proof}\leanok
The numerator is bounded and the denominator tends to infinity.
\end{proof}


We are ready for the first case of the Perron formula, namely when $x<1$:
\begin{lemma}[formulaLtOne]\label{formulaLtOne}\lean{Perron.formulaLtOne}\leanok
For $x>0$, $\sigma>0$, and $x<1$, we have
$$
\frac1{2\pi i}
\int_{(\sigma)}\frac{x^s}{s(s+1)}ds =0.
$$
\end{lemma}


\begin{proof}\leanok
\uses{isHolomorphicOn, HolomorphicOn.vanishesOnRectangle, integralPosAux,
vertIntBound, limitOfConstant,
tendsto_rpow_atTop_nhds_zero_of_norm_lt_one,
tendsto_zero_Lower, tendsto_zero_Upper, isIntegrable}
  Let $f(s) = x^s/(s(s+1))$. Then $f$ is holomorphic on the half-plane $\{s\in\mathbb{C}:\Re(s)>0\}$.
  The rectangle integral of $f$ with corners $\sigma-iT$ and $\sigma+iT$ is zero.
  The limit of this rectangle integral as $T\to\infty$ is $\int_{(\sigma')}-\int_{(\sigma)}$.
  Therefore, $\int_{(\sigma')}=\int_{(\sigma)}$.

 But we also have the bound $\int_{(\sigma')} \leq x^{\sigma'} * C$, where

 $C=\int_\R\frac{1}{|(1+t)(1+t+1)|}dt$.

 Therefore $\int_{(\sigma')}\to 0$ as $\sigma'\to\infty$.

\end{proof}


The second case is when $x>1$.
Here are some auxiliary lemmata for the second case.
TODO: Move to more general section


\begin{lemma}[keyIdentity]\label{keyIdentity}\lean{Perron.keyIdentity}\leanok
Let $x\in \R$ and $s \ne 0, -1$. Then
$$
\frac{x^\sigma}{s(1+s)} = \frac{x^\sigma}{s} - \frac{x^\sigma}{1+s}
$$
\end{lemma}


\begin{proof}\leanok
By ring.
\end{proof}


\begin{lemma}[diffBddAtZero]\label{diffBddAtZero}\lean{Perron.diffBddAtZero}\leanok
Let $x>0$. Then for $0 < c < 1 /2$, we have that the function
$$
s ↦ \frac{x^s}{s(s+1)} - \frac1s
$$
is bounded above on the rectangle with corners at $-c-i*c$ and $c+i*c$ (except at $s=0$).
\end{lemma}


\begin{proof}\uses{keyIdentity}\leanok
Applying Lemma \ref{keyIdentity}, the
 function $s ↦ x^s/s(s+1) - 1/s = x^s/s - x^0/s - x^s/(1+s)$. The last term is bounded for $s$
 away from $-1$. The first two terms are the difference quotient of the function $s ↦ x^s$ at
 $0$; since it's differentiable, the difference remains bounded as $s\to 0$.
\end{proof}


\begin{lemma}[diffBddAtNegOne]\label{diffBddAtNegOne}\lean{Perron.diffBddAtNegOne}\leanok
Let $x>0$. Then for $0 < c < 1 /2$, we have that the function
$$
s ↦ \frac{x^s}{s(s+1)} - \frac{-x^{-1}}{s+1}
$$
is bounded above on the rectangle with corners at $-1-c-i*c$ and $-1+c+i*c$ (except at $s=-1$).
\end{lemma}


\begin{proof}\uses{keyIdentity}\leanok
Applying Lemma \ref{keyIdentity}, the
 function $s ↦ x^s/s(s+1) - x^{-1}/(s+1) = x^s/s - x^s/(s+1) - (-x^{-1})/(s+1)$. The first term is bounded for $s$
 away from $0$. The last two terms are the difference quotient of the function $s ↦ x^s$ at
 $-1$; since it's differentiable, the difference remains bounded as $s\to -1$.
\end{proof}


\begin{lemma}[residueAtZero]\label{residueAtZero}\lean{Perron.residueAtZero}\leanok
Let $x>0$. Then for all sufficiently small $c>0$, we have that
$$
\frac1{2\pi i}
\int_{-c-i*c}^{c+ i*c}\frac{x^s}{s(s+1)}ds = 1.
$$
\end{lemma}


\begin{proof}\leanok
\uses{diffBddAtZero, ResidueTheoremOnRectangleWithSimplePole,
existsDifferentiableOn_of_bddAbove}
For $c>0$ sufficiently small,

 $x^s/(s(s+1))$ is equal to $1/s$ plus a function, $g$, say,
holomorphic in the whole rectangle (by Lemma \ref{diffBddAtZero}).

 Now apply Lemma \ref{ResidueTheoremOnRectangleWithSimplePole}.

\end{proof}


\begin{lemma}[residuePull1]\label{residuePull1}\lean{Perron.residuePull1}\leanok
For $x>1$ (of course $x>0$ would suffice) and $\sigma>0$, we have
$$
\frac1{2\pi i}
\int_{(\sigma)}\frac{x^s}{s(s+1)}ds =1
+
\frac 1{2\pi i}
\int_{(-1/2)}\frac{x^s}{s(s+1)}ds.
$$
\end{lemma}


\begin{proof}\leanok
\uses{residueAtZero}
We pull to a square with corners at $-c-i*c$ and $c+i*c$ for $c>0$
sufficiently small.
By Lemma \ref{residueAtZero}, the integral over this square is equal to $1$.
\end{proof}


\begin{lemma}[residuePull2]\label{residuePull2}\lean{Perron.residuePull2}\leanok
For $x>1$, we have
$$
\frac1{2\pi i}
\int_{(-1/2)}\frac{x^s}{s(s+1)}ds = -1/x +
\frac 1{2\pi i}
\int_{(-3/2)}\frac{x^s}{s(s+1)}ds.
$$
\end{lemma}


\begin{proof}\leanok
\uses{residueAtNegOne}
Pull contour from $(-1/2)$ to $(-3/2)$.
\end{proof}


\begin{lemma}[contourPull3]\label{contourPull3}\lean{Perron.contourPull3}\leanok
For $x>1$ and $\sigma<-3/2$, we have
$$
\frac1{2\pi i}
\int_{(-3/2)}\frac{x^s}{s(s+1)}ds = \frac 1{2\pi i}
\int_{(\sigma)}\frac{x^s}{s(s+1)}ds.
$$
\end{lemma}


\begin{proof}\leanok
Pull contour from $(-3/2)$ to $(\sigma)$.
\end{proof}


\begin{lemma}[formulaGtOne]\label{formulaGtOne}\lean{Perron.formulaGtOne}\leanok
For $x>1$ and $\sigma>0$, we have
$$
\frac1{2\pi i}
\int_{(\sigma)}\frac{x^s}{s(s+1)}ds =1-1/x.
$$
\end{lemma}


\begin{proof}\leanok
\uses{isHolomorphicOn, residuePull1,
residuePull2, contourPull3, integralPosAux, vertIntBoundLeft,
tendsto_rpow_atTop_nhds_zero_of_norm_gt_one, limitOfConstantLeft}
  Let $f(s) = x^s/(s(s+1))$. Then $f$ is holomorphic on $\C \setminus {0,1}$.

 First pull the contour from $(\sigma)$ to $(-1/2)$, picking up a residue $1$ at $s=0$.

 Next pull the contour from $(-1/2)$ to $(-3/2)$, picking up a residue $-1/x$ at $s=-1$.

 Then pull the contour all the way to $(\sigma')$ with $\sigma'<-3/2$.

 For $\sigma' < -3/2$, the integral is bounded by $x^{\sigma'}\int_\R\frac{1}{|(1+t^2)(2+t^2)|^{1/2}}dt$.

 Therefore $\int_{(\sigma')}\to 0$ as $\sigma'\to\infty$.


\end{proof}


The two together give the Perron formula. (Which doesn't need to be a separate lemma.)

For $x>0$ and $\sigma>0$, we have
$$
\frac1{2\pi i}
\int_{(\sigma)}\frac{x^s}{s(s+1)}ds = \begin{cases}
1-\frac1x & \text{ if }x>1\\
0 & \text{ if } x<1
\end{cases}.
$$



\section{Mellin transforms}

In this section, we define the Mellin transform (already in Mathlib, thanks to David Loeffler), prove its inversion formula, and
derive a number of important properties of some special functions and bumpfunctions.

\begin{definition}\label{MellinTransform}
Let $f$ be a function from $\mathbb{R}_{>0}$ to $\mathbb{C}$. We define the Mellin transform of $f$ to be the function $\mathcal{M}(f)$ from $\mathbb{C}$ to $\mathbb{C}$ defined by
$$\mathcal{M}(f)(s) = \int_0^\infty f(x)x^{s-1}dx.$$
\end{definition}

[Note: My preferred way to think about this is that we are integrating over the multiplicative group $\mathbb{R}_{>0}$, multiplying by a (not necessarily unitary!) character $|\cdot|^s$, and integrating with respect to the invariant Haar measure $dx/x$. This is very useful in the kinds of calculations carried out below. But may be more difficult to formalize as things now stand. So we
might have clunkier calculations, which ``magically'' turn out just right - of course they're explained by the aforementioned structure...]




It is very convenient to define integrals along vertical lines in the complex plane, as follows.
\begin{definition}\label{VerticalIntegral}
Let $f$ be a function from $\mathbb{C}$ to $\mathbb{C}$, and let $\sigma$ be a real number. Then we define
$$\int_{(\sigma)}f(s)ds = \int_{\sigma-i\infty}^{\sigma+i\infty}f(s)ds.$$
\end{definition}
[Note: Better to define $\int_{(\sigma)}$ as $\frac1{2\pi i}\int_{\sigma-i\infty}^{\sigma+i\infty}$??
There's a factor of $2\pi i$ in such contour integrals...]



We first prove the following ``Perron-type'' formula.
\begin{lemma}\label{PerronFormula}
For $x>0$ and $\sigma>1$, we have
$$
\frac1{2\pi i}
\int_{(\sigma)}\frac{x^s}{s(s+1)}ds = \begin{cases}
1-\frac1x & \text{ if }x>1\\
0 & \text{ if } x<1
\end{cases}.
$$
\end{lemma}



\begin{proof}
\uses{ResidueTheoremOnRectangle, RectangleIntegralEqSumOfRectangles, VerticalIntegral, MellinTransform}
Pull contours and collect residues. This only involves rectangles, and everything is absolutely convergent.
\end{proof}



\begin{theorem}\label{MellinInversion}
Let $f$ be a nice function from $\mathbb{R}_{>0}$ to $\mathbb{C}$, and let $\sigma$ be sufficiently large. Then
$$f(x) = \frac{1}{2\pi i}\int_{(\sigma)}\mathcal{M}(f)(s)x^{-s}ds.$$
\end{theorem}

[Note: How ``nice''? Schwartz (on $(0,\infty)$) is certainly enough. As we formalize this, we can add whatever conditions are necessary for the proof to go through.]



\begin{proof}
\uses{PerronFormula}
The proof is from [Goldfeld-Kontorovich 2012].
Integrate by parts twice.
$$
\mathcal{M}(f)(s) = \int_0^\infty f(x)x^{s-1}dx = - \int_0^\infty f'(x)x^s\frac{1}{s}dx = \int_0^\infty f''(x)x^{s+1}\frac{1}{s(s+1)}dx.
$$
Assuming $f$ is Schwartz, say, we now have at least quadratic decay in $s$ of the Mellin transform. Inserting this formula into the inversion formula and Fubini-Tonelli (we now have absolute convergence!) gives:
$$
RHS = \frac{1}{2\pi i}\left(\int_{(\sigma)}\int_0^\infty f''(t)t^{s+1}\frac{1}{s(s+1)}dt\right) x^{-s}ds
$$
$$
= \int_0^\infty f''(t) t \left( \frac{1}{2\pi i}\int_{(\sigma)}(t/x)^s\frac{1}{s(s+1)}ds\right) dt.
$$
Apply the Perron formula to the inside:
$$
= \int_x^\infty f''(t) t \left(1-\frac{x}{t}\right)dt
= -\int_x^\infty f'(t) dt
= f(x),
$$
where we integrated by parts (undoing the first partial integration), and finally applied the fundamental theorem of calculus (undoing the second).
\end{proof}



Finally, we need Mellin Convolutions and properties thereof.
\begin{definition}\label{MellinConvolution}
Let $f$ and $g$ be functions from $\mathbb{R}_{>0}$ to $\mathbb{C}$. Then we define the Mellin convolution of $f$ and $g$ to be the function $f\ast g$ from $\mathbb{R}_{>0}$ to $\mathbb{C}$ defined by
$$(f\ast g)(x) = \int_0^\infty f(y)g(x/y)\frac{dy}{y}.$$
\end{definition}



The Mellin transform of a convolution is the product of the Mellin transforms.
\begin{theorem}\label{MellinConvolutionTransform}
Let $f$ and $g$ be functions from $\mathbb{R}_{>0}$ to $\mathbb{C}$. Then
$$\mathcal{M}(f\ast g)(s) = \mathcal{M}(f)(s)\mathcal{M}(g)(s).$$
\end{theorem}



\begin{proof}
\uses{MellinTransform}
This is a straightforward calculation.
\end{proof}



Let $\psi$ be a bumpfunction.
\begin{theorem}\label{SmoothExistence}
There exists a smooth (once differentiable would be enough), nonnegative ``bumpfunction'' $\psi$,
 supported in $[1/2,2]$ with total mass one:
$$
\int_0^\infty \psi(x)\frac{dx}{x} = 1.
$$
\end{theorem}
\begin{proof}
\uses{smooth-ury}
Same idea as Urysohn-type argument.
\end{proof}



The $\psi$ function has Mellin transform $\mathcal{M}(\psi)(s)$ which is entire and decays (at least) like $1/|s|$.
\begin{theorem}\label{MellinOfPsi}
The Mellin transform of $\psi$ is
$$\mathcal{M}(\psi)(s) =  O\left(\frac{1}{|s|}\right),$$
as $|s|\to\infty$.
\end{theorem}

[Of course it decays faster than any power of $|s|$, but it turns out that we will just need one power.]



\begin{proof}
\uses{MellinTransform, SmoothExistence}
Integrate by parts once.
\end{proof}



We can make a delta spike out of this bumpfunction, as follows.
\begin{definition}\label{DeltaSpike}
\uses{SmoothExistence}
Let $\psi$ be a bumpfunction supported in $[1/2,2]$. Then for any $\epsilon>0$, we define the delta spike $\psi_\epsilon$ to be the function from $\mathbb{R}_{>0}$ to $\mathbb{C}$ defined by
$$\psi_\epsilon(x) = \frac{1}{\epsilon}\psi\left(x^{\frac{1}{\epsilon}}\right).$$
\end{definition}

This spike still has mass one:
\begin{lemma}\label{DeltaSpikeMass}
For any $\epsilon>0$, we have
$$\int_0^\infty \psi_\epsilon(x)\frac{dx}{x} = 1.$$
\end{lemma}



\begin{proof}
\uses{DeltaSpike}
Substitute $y=x^{1/\epsilon}$, and use the fact that $\psi$ has mass one, and that $dx/x$ is Haar measure.
\end{proof}



The Mellin transform of the delta spike is easy to compute.
\begin{theorem}\label{MellinOfDeltaSpike}
For any $\epsilon>0$, the Mellin transform of $\psi_\epsilon$ is
$$\mathcal{M}(\psi_\epsilon)(s) = \mathcal{M}(\psi)\left(\epsilon s\right).$$
\end{theorem}



\begin{proof}
\uses{DeltaSpike, MellinTransform}
Substitute $y=x^{1/\epsilon}$, use Haar measure; direct calculation.
\end{proof}



In particular, for $s=1$, we have that the Mellin transform of $\psi_\epsilon$ is $1+O(\epsilon)$.
\begin{corollary}\label{MellinOfDeltaSpikeAt1}
For any $\epsilon>0$, we have
$$\mathcal{M}(\psi_\epsilon)(1) =
\mathcal{M}(\psi)(\epsilon)= 1+O(\epsilon).$$
\end{corollary}



\begin{proof}
\uses{MellinOfDeltaSpike, DeltaSpikeMass}
This is immediate from the above theorem, the fact that $\mathcal{M}(\psi)(0)=1$ (total mass one),
and that $\psi$ is Lipschitz.
\end{proof}



Let $1_{(0,1]}$ be the function from $\mathbb{R}_{>0}$ to $\mathbb{C}$ defined by
$$1_{(0,1]}(x) = \begin{cases}
1 & \text{ if }x\leq 1\\
0 & \text{ if }x>1
\end{cases}.$$
This has Mellin transform
\begin{theorem}\label{MellinOf1}
The Mellin transform of $1_{(0,1]}$ is
$$\mathcal{M}(1_{(0,1]})(s) = \frac{1}{s}.$$
\end{theorem}
[Note: this already exists in mathlib]



What will be essential for us is properties of the smooth version of $1_{(0,1]}$, obtained as the
 Mellin convolution of $1_{(0,1]}$ with $\psi_\epsilon$.
\begin{definition}\label{Smooth1}\uses{MellinOf1, MellinConvolution}
Let $\epsilon>0$. Then we define the smooth function $\widetilde{1_{\epsilon}}$ from $\mathbb{R}_{>0}$ to $\mathbb{C}$ by
$$\widetilde{1_{\epsilon}} = 1_{(0,1]}\ast\psi_\epsilon.$$
\end{definition}



In particular, we have the following
\begin{lemma}\label{Smooth1Properties}
Fix $\epsilon>0$. There is an absolute constant $c>0$ so that:

(1) If $x\leq (1-c\epsilon)$, then
$$\widetilde{1_{\epsilon}}(x) = 1.$$

And (2):
if $x\geq (1+c\epsilon)$, then
$$\widetilde{1_{\epsilon}}(x) = 0.$$
\end{lemma}



\begin{proof}
\uses{Smooth1, MellinConvolution}
This is a straightforward calculation, using the fact that $\psi_\epsilon$ is supported in $[1/2^\epsilon,2^\epsilon]$.
\end{proof}



Combining the above, we have the following Main Lemma of this section on the Mellin transform of $\widetilde{1_{\epsilon}}$.
\begin{lemma}\label{MellinOfSmooth1}\uses{Smooth1Properties, MellinConvolutionTransform, MellinOfDeltaSpikeAt1}
Fix  $\epsilon>0$. Then the Mellin transform of $\widetilde{1_{\epsilon}}$ is
$$\mathcal{M}(\widetilde{1_{\epsilon}})(s) = \frac{1}{s}\left(\mathcal{M}(\psi)\left(\epsilon s\right)\right).$$

For any $s$, we have the bound
$$\mathcal{M}(\widetilde{1_{\epsilon}})(s) = O\left(\frac{1}{\epsilon|s|^2}\right).$$

At $s=1$, we have
$$\mathcal{M}(\widetilde{1_{\epsilon}})(1) = (1+O(\epsilon)).$$
\end{lemma}




\section{Zeta Bounds}

\begin{lemma}[sum_eq_int_deriv_aux]\label{sum_eq_int_deriv_aux}\lean{sum_eq_int_deriv_aux}\leanok
  Let $k \le a < b\le k+1$, with $k$ an integer, and let $\phi$ be continuously differentiable on
  $[a, b]$.
  Then
  \[
  \sum_{a < n \le b} \phi(n) = \int_a^b \phi(x) \, dx + \left(\lfloor b \rfloor + \frac{1}{2} - b\right) \phi(b) - \left(\lfloor a \rfloor + \frac{1}{2} - a\right) \phi(a) - \int_a^b \left(\lfloor x \rfloor + \frac{1}{2} - x\right) \phi'(x) \, dx.
  \]
\end{lemma}


\begin{proof}\leanok
Partial integration.
\end{proof}


\begin{lemma}[sum_eq_int_deriv]\label{sum_eq_int_deriv}\lean{sum_eq_int_deriv}\leanok
  Let $a < b$, and let $\phi$ be continuously differentiable on $[a, b]$.
  Then
  \[
  \sum_{a < n \le b} \phi(n) = \int_a^b \phi(x) \, dx + \left(\lfloor b \rfloor + \frac{1}{2} - b\right) \phi(b) - \left(\lfloor a \rfloor + \frac{1}{2} - a\right) \phi(a) - \int_a^b \left(\lfloor x \rfloor + \frac{1}{2} - x\right) \phi'(x) \, dx.
  \]
\end{lemma}


\begin{proof}\uses{sum_eq_int_deriv_aux}
  Apply Lemma \ref{sum_eq_int_deriv_aux} in blocks of length $\le 1$.
\end{proof}


\begin{lemma}[ZetaSum_aux1]\label{ZetaSum_aux1}\lean{ZetaSum_aux1}\leanok
  Let $a < b$ be natural numbers and $s\in \C$ with $s \ne 1$.
  Then
  \[
  \sum_{a < n \le b} \frac{1}{n^s} =  \frac{b^{1-s} - a^{1-s}}{1-s} + \frac{b^{-s}-a^{-s}}{2} + s \int_a^b \frac{\lfloor x\rfloor + 1/2 - x}{x^{s+1}} \, dx.
  \]
\end{lemma}


\begin{proof}\uses{sum_eq_int_deriv}
  Apply Lemma \ref{sum_eq_int_deriv} to the function $x \mapsto x^{-s}$.
\end{proof}


\begin{lemma}[ZetaSum_aux1a]\label{ZetaSum_aux1a}\lean{ZetaSum_aux1a}\leanok
For any $0 < a < b$ and  $s \in \C$ with $\sigma=\Re(s)>0$,
$$
\left|\int_a^b \frac{\lfloor x\rfloor + 1/2 - x}{x^{s+1}} \, dx\right|
\le \frac{a^{-\sigma}-b^{-\sigma}}/{\sigma}.
$$
\end{lemma}


\begin{proof}
Apply the triangle inequality
$$
\left|\int_a^b \frac{\lfloor x\rfloor + 1/2 - x}{x^{s+1}} \, dx\right|
\le \int_a^b \frac{1}{x^{\sigma+1}} \, dx,
$$
and evaluate the integral.
\end{proof}


\begin{lemma}[ZetaSum_aux2]\label{ZetaSum_aux2}\lean{ZetaSum_aux2}\leanok
  Let $N$ be a natural number and $s\in \C$, $\Re(s)>1$.
  Then
  \[
  \sum_{N < n} \frac{1}{n^s} =  \frac{- N^{1-s}}{1-s} + \frac{-N^{-s}}{2} + s \int_N^\infty \frac{\lfloor x\rfloor + 1/2 - x}{x^{s+1}} \, dx.
  \]
\end{lemma}


\begin{proof}\uses{ZetaSum_aux1, ZetaSum_aux1a}
  Apply Lemma \ref{ZetaSum_aux1} with $a=N$ and $b\to \infty$.
\end{proof}


\begin{definition}[RiemannZeta0]\label{RiemannZeta0}\lean{RiemannZeta0}\leanok
\uses{ZetaSum_aux2}
For any natural $N\ge1$, we define
$$
\zeta_0(N,s) :=
\sum_{1\le n < N} \frac1{n^s}
+
\frac{- N^{1-s}}{1-s} + \frac{-N^{-s}}{2} + s \int_N^\infty \frac{\lfloor x\rfloor + 1/2 - x}{x^{s+1}} \, dx
$$
\end{definition}


\begin{lemma}[ZetaBndAux]\label{ZetaBndAux}\lean{ZetaBndAux}\leanok
For any $N\ge1$ and $s\in \C$, $\sigma=\Re(s)\in[1/2,2]$,
$$
s\int_N^\infty \frac{\lfloor x\rfloor + 1/2 - x}{x^{s+1}} \, dx
\ll |t| \frac{N^{-\sigma}}{\sigma},
$$
as $|t|\to\infty$.
\end{lemma}


\begin{proof}\uses{ZetaSum_aux1a}
Apply Lemma \ref{ZetaSum_aux1a} with $a=N$ and $b\to \infty$, and estimate $|s|\ll |t|$.
\end{proof}


\begin{lemma}[Zeta0EqZeta]\label{Zeta0EqZeta}\lean{Zeta0EqZeta}\leanok
If $\Re(s)>0$, then for any $N$,
$$
\zeta_0(N,s) = \zeta(s).
$$
[** What about junk values at $s=1$? Maybe add $s\ne1$. **]
\end{lemma}


\begin{proof}
\uses{ZetaSum_aux2, RiemannZeta0, ZetaBnd_aux1}
Use Lemma \ref{ZetaSum_aux2} and the Definition \ref{RiemannZeta0}.
\end{proof}


\begin{lemma}[ZetaBnd_aux2]\label{ZetaBnd_aux2}\lean{ZetaBnd_aux2}\leanok
Given $n ≤ t$ and $\sigma$ with $1-A/\log t \le \sigma$, we have
that
$$
|n^{-s}| \le n^{-1} e^A.
$$
\end{lemma}


\begin{proof}
Use $|n^{-s}| = n^{-\sigma}
= e^{-\sigma \log n}
\le
\exp(-\left(1-\frac{A}{\log t}\right)\log n)
\le
n^{-1} e^A$,
since $n\le t$.
\end{proof}


\begin{lemma}[ZetaUpperBnd]\label{ZetaUpperBnd}\lean{ZetaUpperBnd}\leanok
For any $s\in \C$, $1/2 \le \Re(s)=\sigma\le 2$,
and any $A>0$ sufficiently small, and $1-A/\log t \le \sigma$, we have
$$
|\zeta(s)| \ll \log t,
$$
as $|t|\to\infty$.
\end{lemma}


\begin{proof}\uses{ZetaBnd_aux1, ZetaBnd_aux2}
First replace $\zeta(s)$ by $\zeta_0(N,s)$ for $N = \lfloor |t| \rfloor$.
We estimate:
$$
|\zeta_0(N,s)| \ll
\sum_{1\le n < |t|} |n^{-s}|
+
\frac{- |t|^{1-\sigma}}{|1-s|} + \frac{-|t|^{-\sigma}}{2} +
|t| * |t| ^ (-σ) / σ
$$
$$
\ll
e^A \sum_{1\le n < |t|} n^{-1}
+|t|^{1-\sigma}
$$
,
where we used Lemma \ref{ZetaBnd_aux2} and Lemma \ref{ZetaBnd_aux1}.
The first term is $\ll \log |t|$.
For the second term, estimate
$$
|t|^{1-\sigma}
\le |t|^{1-(1-A/\log |t|)}
= |t|^{A/\log |t|} \ll 1.
$$
\end{proof}


\begin{lemma}[ZetaDerivUpperBnd]\label{ZetaDerivUpperBnd}\lean{ZetaDerivUpperBnd}\leanok
For any $s\in \C$, $1/2 \le \Re(s)=\sigma\le 2$,
and any $A>0$ sufficiently small, and $1-A/\log t \le \sigma$, we have
$$
|\zeta'(s)| \ll \log^2 t,
$$
as $|t|\to\infty$.
\end{lemma}


\begin{proof}\uses{ZetaBnd_aux1, ZetaBnd_aux2}
First replace $\zeta(s)$ by $\zeta_0(N,s)$ for $N = \lfloor |t| \rfloor$.
Differentiating term by term, we get:
$$
\zeta'(s) = -\sum_{1\le n < N} n^{-s} \log n
-
\frac{N^{1 - s}}{1 - s)^2} + \frac{N^{1 - s} \log N} {1 - s}
+ \frac{-N^{-s}\log N}{2} +
\int_N^\infty \frac{\lfloor x\rfloor + 1/2 - x}{x^{s+1}} \, dx
-
s(s+1) \int_N^\infty \frac{\lfloor x\rfloor + 1/2 - x}{x^{s+2}} \, dx
.
$$
Estimate as before, with an extra factor of $\log |t|$.
\end{proof}


\begin{lemma}[ZetaNear1Bnd]\label{ZetaNear1Bnd}\lean{ZetaNear1Bnd}\leanok
As $\simga\to1^+$,
$$
|\zeta(\sigma)| \ll (\sigma-1).
$$
\end{lemma}


\begin{proof}\uses{ZetaBnd_aux1, Zeta0EqZeta}
Zeta has a simple pole at $s=1$. Equivalently, $\zeta(s)(s-1)$ remains bounded near $1$.
Lots of ways to prove this.
Probably the easiest one: use the expression for $\zeta_0 (N,s)$ with $N=1$ (the term $N^{1-s}/(1-s)$ being the only unbounded one).
\end{proof}


\begin{lemma}[ZetaInvBound1]\label{ZetaInvBound1}\lean{ZetaInvBound1}\leanok
For all $\sigma>1$,
$$
1/|\zeta(\sigma+it)| \le |\zeta(\sigma)|^{3/4}|\zeta(\sigma+2it)|^{1/4}
$$
\end{lemma}


\begin{proof}
The identity
$$
1 \le |\zeta(\sigma)|^3 |\zeta(\sigma+it)|^4 |\zeta(\sigma+2it)|
$$
for $\sigma>1$
is already proved by Michael Stoll in the EulerProducts PNT file.
\end{proof}


\begin{lemma}[ZetaInvBound2]\label{ZetaInvBound2}\lean{ZetaInvBound2}\leanok
For $\sigma>1$ (and $\sigma \le 2$),
$$
1/|\zeta(\sigma+it)| \ll (\sigma-1)^{3/4}(\log |t|)^{1/4},
$$
as $|t|\to\infty$.
\end{lemma}


\begin{proof}\uses{ZetaInvBound1, ZetaNear1Bnd, ZetaUpperBnd}
Combine Lemma \ref{ZetaInvBound1} with the bounds in Lemmata \ref{ZetaNear1Bnd} and
\ref{ZetaUpperBnd}.


\begin{lemma}[Zeta_eq_int_derivZeta]\label{Zeta_eq_int_derivZeta}\lean{Zeta_eq_int_derivZeta}
\leanok
For any $t\ne0$ (so we don't pass through the pole), and $\sigma_1 < \sigma_2$,
$$
\int_{\sigma_1}^{\sigma_2}\zeta'(\sigma + it) dt =
\zeta(\sigma_2+it) - \zeta(\sigma_1+it).
$$
\end{lemma}


\begin{proof}
This is the fundamental theorem of calculus.
\end{proof}


\begin{lemma}[Zeta_diff_Bnd]\label{Zeta_diff_Bnd}\lean{Zeta_diff_Bnd}\leanok
For any $A>0$ sufficiently small, there is a constant $C>0$ so that
whenever $1- A / \log t \le \sigma_1, \sigma_2\le 2$, we have that:
$$
|\zeta (\sigma_2 + it) - \zeta (\sigma_1 + it)|
\le C (\log |t|)^2 (\sigma_2 - \sigma_1).
$$
\end{lemma}


\begin{proof}
\uses{Zeta_eq_int_derivZeta, ZetaDerivUpperBnd}
Use Lemma \ref{Zeta_eq_int_derivZeta} and
estimate trivially using Lemma \ref{ZetaDerivUpperBnd}.
\end{proof}


\begin{lemma}[ZetaInvBnd]\label{ZetaInvBnd}\lean{ZetaInvBnd}\leanok
Lemma.
\end{lemma}


\begin{proof}
\uses{Zeta_diff_Bnd}
Proof.
\end{proof}



\section{Proof of Medium PNT}
\input{MediumPNT.tex}


%\chapter{Third Approach}

%\section{Hadamard factorization}
%

In this file, we prove the Hadamard Factorization theorem for functions of finite order, and prove that the zeta function
is such.




%\section{Hoffstein-Lockhart}
%

In this file, we use the Hoffstein-Lockhart construction to prove a zero-free region for zeta.

ZeroFreeRegion

Hoffstein-Lockhart + Goldfeld-Hoffstein-Liemann





\section{Strong PNT}

\begin{definition}\label{ChebyshevPsi}\lean{ChebyshevPsi}\leanok
The Chebyshev Psi function is defined as
$$
\psi(x) = \sum_{n \leq x} \Lambda(n),
$$
where $\Lambda(n)$ is the von Mangoldt function.
\end{definition}




Main Theorem: The Prime Number Theorem in strong form.
\begin{theorem}[PrimeNumberTheorem]\label{StrongPNT}\lean{PrimeNumberTheorem}
There is a constant $c > 0$ such that
$$
ψ (x) = x + O(x e^{-c \sqrt{\log x}})
$$
as $x\to \infty$.
\end{theorem}





\chapter{Elementary Corollaries}


\begin{lemma}\label{range-eq-range}\lean{finsum_range_eq_sum_range, finsum_range_eq_sum_range'}\leanok For any arithmetic function $f$ and real number $x$, one has
$$ \sum_{n \leq x} f(n) = \sum_{n \leq ⌊x⌋_+} f(n)$$
and
$$ \sum_{n < x} f(n) = \sum_{n < ⌈x⌉_+} f(n).$$
\end{lemma}


\begin{proof}\leanok Straightforward. \end{proof}


\begin{theorem}\label{chebyshev-asymptotic}\lean{chebyshev_asymptotic}\leanok  One has
  $$ \sum_{p \leq x} \log p = x + o(x).$$
\end{theorem}


\begin{proof}
\uses{WeakPNT, range-eq-range}
From the prime number theorem we already have
$$ \sum_{n \leq x} \Lambda(n) = x + o(x)$$
so it suffices to show that
$$ \sum_{j \geq 2} \sum_{p^j \leq x} \log p = o(x).$$
Only the terms with $j \leq \log x / \log 2$ contribute, and each $j$ contributes at most $\sqrt{x} \log x$ to the sum, so the left-hand side is $O( \sqrt{x} \log^2 x ) = o(x)$ as required.
\end{proof}


\begin{corollary}[Bounds on primorial]  \label{primorial-bounds}\lean{primorial_bounds}\leanok
We have
  $$ \prod_{p \leq x} p = \exp( x + o(x) )$$
\end{corollary}


\begin{proof}
\uses{chebyshev-asymptotic}
  Exponentiate Theorem \ref{chebyshev-asymptotic}.
\end{proof}


Let $\pi(x)$ denote the number of primes up to $x$.

\begin{theorem}\label{pi-asymp}\lean{pi_asymp}\leanok  One has
  $$ \pi(x) = (1+o(1)) \int_2^x \frac{dt}{\log t}$$
as $x \to \infty$.
\end{theorem}


\begin{proof}
\uses{chebyshev-asymptotic}
We have the identity
$$ \pi(x) = \frac{1}{\log x} \sum_{p \leq x} \log p
+ \int_2^x (\sum_{p \leq t} \log p) \frac{dt}{t \log^2 t}$$
as can be proven by interchanging the sum and integral and using the fundamental theorem of calculus.  For any $\eps$, we know from Theorem \ref{chebyshev-asymptotic} that there is $x_\eps$ such that
$\sum_{p \leq t} \log p = t + O(\eps t)$ for $t \geq x_\eps$, hence for $x \geq x_\eps$
$$ \pi(x) = \frac{1}{\log x} (x + O(\eps x))
+ \int_{x_\eps}^x (t + O(\eps t)) \frac{dt}{t \log^2 t} + O_\eps(1)$$
where the $O_\eps(1)$ term can depend on $x_\eps$ but is independent of $x$.  One can evaluate this after an integration by parts as
$$ \pi(x) = (1+O(\eps)) \int_{x_\eps}^x \frac{dt}{\log t} + O_\eps(1)$$
$$  = (1+O(\eps)) \int_{2}^x \frac{dt}{\log t} $$
for $x$ large enough, giving the claim.
\end{proof}


\begin{corollary}\label{pi-alt}\lean{pi_alt}\leanok  One has
$$ \pi(x) = (1+o(1)) \frac{x}{\log x}$$
as $x \to \infty$.
\end{corollary}


\begin{proof}
\uses{pi-asymp}
An integration by parts gives
  $$ \int_2^x \frac{dt}{\log t} = \frac{x}{\log x} - \frac{2}{\log 2} + \int_2^x \frac{dt}{\log^2 t}.$$
We have the crude bounds
$$ \int_2^{\sqrt{x}} \frac{dt}{\log^2 t} = O( \sqrt{x} )$$
and
$$ \int_{\sqrt{x}}^x \frac{dt}{\log^2 t} = O( \frac{x}{\log^2 x} )$$
and combining all this we obtain
$$ \int_2^x \frac{dt}{\log t} = \frac{x}{\log x} + O( \frac{x}{\log^2 x} )$$
$$ = (1+o(1)) \frac{x}{\log x}$$
and the claim then follows from Theorem \ref{pi-asymp}.
\end{proof}


Let $p_n$ denote the $n^{th}$ prime.

\begin{proposition}\label{pn-asymptotic}\lean{pn_asymptotic}\leanok
 One has
  $$ p_n = (1+o(1)) n \log n$$
as $n \to \infty$.
\end{proposition}


\begin{proof}
\uses{pi-alt}
Use Corollary \ref{pi-alt} to show that for any $\eps>0$, and for $x$ sufficiently large, the number of primes up to $(1-\eps) n \log n$ is less than $n$, and the number of primes up to $(1+\eps) n \log n$ is greater than $n$.
\end{proof}


\begin{corollary} \label{pn-pnPlus1}\lean{pn_pn_plus_one}\leanok
We have $p_{n+1} - p_n = o(p_n)$
  as $n \to \infty$.
\end{corollary}


\begin{proof}
\uses{pn-asymptotic}
  Easy consequence of preceding proposition.
\end{proof}


\begin{corollary}  \label{prime-between}\lean{prime_between}\leanok
For every $\eps>0$, there is a prime between $x$ and $(1+\eps)x$ for all sufficiently large $x$.
\end{corollary}


\begin{proof}
\uses{pi-alt}
Use Corollary \ref{pi-alt} to show that $\pi((1+\eps)x) - \pi(x)$ goes to infinity as $x \to \infty$.
\end{proof}


\begin{proposition}\label{mun}\lean{sum_mobius_div_self_le}\leanok
We have $|\sum_{n \leq x} \frac{\mu(n)}{n}| \leq 1$.
\end{proposition}


\begin{proof}
From M\"obius inversion $1_{n=1} = \sum_{d|n} \mu(d)$ and summing we have
  $$ 1 = \sum_{d \leq x} \mu(d) \lfloor \frac{x}{d} \rfloor$$
  for any $x \geq 1$. Since $\lfloor \frac{x}{d} \rfloor = \frac{x}{d} - \epsilon_d$ with
  $0 \leq \epsilon_d < 1$ and $\epsilon_x = 0$, we conclude that
  $$ 1 = x \sum_{d \leq x} \frac{\mu(d)}{d} - (x - 1)$$
  and the claim follows.
\end{proof}


\begin{proposition}\label{mu-pnt}\lean{mu_pnt}\leanok  We have $\sum_{n \leq x} \mu(n) = o(x)$.
\end{proposition}


\begin{proof}
\uses{mun, WeakPNT}
From the Dirichlet convolution identity
  $$ \mu(n) \log n = - \sum_{d|n} \mu(d) \Lambda(n/d)$$
and summing we obtain
$$ \sum_{n \leq x} \mu(n) \log n = - \sum_{d \leq x} \mu(d) \sum_{m \leq x/d} \Lambda(m).$$
For any $\eps>0$, we have from the prime number theorem that
$$ \sum_{m \leq x/d} \Lambda(m) = x/d + O(\eps x/d) + O_\eps(1)$$
(divide into cases depending on whether $x/d$ is large or small compared to $\eps$).
We conclude that
$$ \sum_{n \leq x} \mu(n) \log n = - x \sum_{d \leq x} \frac{\mu(d)}{d} + O(\eps x \log x) + O_\eps(x).$$
Applying \eqref{mun} we conclude that
$$ \sum_{n \leq x} \mu(n) \log n = O(\eps x \log x) + O_\eps(x).$$
and hence
$$ \sum_{n \leq x} \mu(n) \log x = O(\eps x \log x) + O_\eps(x) + O( \sum_{n \leq x} (\log x - \log n) ).$$
From Stirling's formula one has
$$  \sum_{n \leq x} (\log x - \log n) = O(x)$$
thus
$$ \sum_{n \leq x} \mu(n) \log x = O(\eps x \log x) + O_\eps(x)$$
and thus
$$ \sum_{n \leq x} \mu(n) = O(\eps x) + O_\eps(\frac{x}{\log x}).$$
Sending $\eps \to 0$ we obtain the claim.
\end{proof}


\begin{proposition}\label{lambda-pnt}\lean{lambda_pnt}\leanok
We have $\sum_{n \leq x} \lambda(n) = o(x)$.
\end{proposition}


\begin{proof}
\uses{mu-pnt}
From the identity
  $$ \lambda(n) = \sum_{d^2|n} \mu(n/d^2)$$
and summing, we have
$$ \sum_{n \leq x} \lambda(n) = \sum_{d \leq \sqrt{x}} \sum_{n \leq x/d^2} \mu(n).$$
For any $\eps>0$, we have from Proposition \ref{mu-pnt} that
$$ \sum_{n \leq x/d^2} \mu(n) = O(\eps x/d^2) + O_\eps(1)$$
and hence on summing in $d$
$$ \sum_{n \leq x} \lambda(n) = O(\eps x) + O_\eps(x^{1/2}).$$
Sending $\eps \to 0$ we obtain the claim.
\end{proof}




\end{document}
