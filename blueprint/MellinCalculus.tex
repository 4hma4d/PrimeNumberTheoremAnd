
In this section, we define the Mellin transform (already in Mathlib, thanks to David Loeffler), prove its inversion formula, and
derive a number of important properties of some special functions and bumpfunctions.

\begin{definition}\label{MellinTransform}
Let $f$ be a function from $\mathbb{R}_{>0}$ to $\mathbb{C}$. We define the Mellin transform of $f$ to be the function $\mathcal{M}(f)$ from $\mathbb{C}$ to $\mathbb{C}$ defined by
$$\mathcal{M}(f)(s) = \int_0^\infty f(x)x^{s-1}dx.$$
\end{definition}

[Note: My preferred way to think about this is that we are integrating over the multiplicative group $\mathbb{R}_{>0}$, multiplying by a (not necessarily unitary!) character $|\cdot|^s$, and integrating with respect to the invariant Haar measure $dx/x$. This is very useful in the kinds of calculations carried out below. But may be more difficult to formalize as things now stand. So we
might have clunkier calculations, which ``magically'' turn out just right - of course they're explained by the aforementioned structure...]




It is very convenient to define integrals along vertical lines in the complex plane, as follows.
\begin{definition}\label{VerticalIntegral}
Let $f$ be a function from $\mathbb{C}$ to $\mathbb{C}$, and let $\sigma$ be a real number. Then we define
$$\int_{(\sigma)}f(s)ds = \int_{\sigma-i\infty}^{\sigma+i\infty}f(s)ds.$$
\end{definition}
[Note: Better to define $\int_{(\sigma)}$ as $\frac1{2\pi i}\int_{\sigma-i\infty}^{\sigma+i\infty}$??
There's a factor of $2\pi i$ in such contour integrals...]



We first prove the following ``Perron-type'' formula.
\begin{lemma}\label{PerronFormula}
For $x>0$ and $\sigma>1$, we have
$$
\frac1{2\pi i}
\int_{(\sigma)}\frac{x^s}{s(s+1)}ds = \begin{cases}
1-\frac1x & \text{ if }x>1\\
0 & \text{ if } x<1
\end{cases}.
$$
\end{lemma}



\begin{proof}
\uses{ResidueTheoremOnRectangle, RectangleIntegralEqSumOfRectangles, VerticalIntegral, MellinTransform}
Pull contours and collect residues. This only involves rectangles, and everything is absolutely convergent.
\end{proof}



\begin{theorem}\label{MellinInversion}
Let $f$ be a nice function from $\mathbb{R}_{>0}$ to $\mathbb{C}$, and let $\sigma$ be sufficiently large. Then
$$f(x) = \frac{1}{2\pi i}\int_{(\sigma)}\mathcal{M}(f)(s)x^{-s}ds.$$
\end{theorem}

[Note: How ``nice''? Schwartz (on $(0,\infty)$) is certainly enough. As we formalize this, we can add whatever conditions are necessary for the proof to go through.]



\begin{proof}
\uses{PerronFormula}
The proof is from [Goldfeld-Kontorovich 2012].
Integrate by parts twice.
$$
\mathcal{M}(f)(s) = \int_0^\infty f(x)x^{s-1}dx = - \int_0^\infty f'(x)x^s\frac{1}{s}dx = \int_0^\infty f''(x)x^{s+1}\frac{1}{s(s+1)}dx.
$$
Assuming $f$ is Schwartz, say, we now have at least quadratic decay in $s$ of the Mellin transform. Inserting this formula into the inversion formula and Fubini-Tonelli (we now have absolute convergence!) gives:
$$
RHS = \frac{1}{2\pi i}\left(\int_{(\sigma)}\int_0^\infty f''(t)t^{s+1}\frac{1}{s(s+1)}dt\right) x^{-s}ds
$$
$$
= \int_0^\infty f''(t) t \left( \frac{1}{2\pi i}\int_{(\sigma)}(t/x)^s\frac{1}{s(s+1)}ds\right) dt.
$$
Apply the Perron formula to the inside:
$$
= \int_x^\infty f''(t) t \left(1-\frac{x}{t}\right)dt
= -\int_x^\infty f'(t) dt
= f(x),
$$
where we integrated by parts (undoing the first partial integration), and finally applied the fundamental theorem of calculus (undoing the second).
\end{proof}



Finally, we need Mellin Convolutions and properties thereof.
\begin{definition}\label{MellinConvolution}
Let $f$ and $g$ be functions from $\mathbb{R}_{>0}$ to $\mathbb{C}$. Then we define the Mellin convolution of $f$ and $g$ to be the function $f\ast g$ from $\mathbb{R}_{>0}$ to $\mathbb{C}$ defined by
$$(f\ast g)(x) = \int_0^\infty f(y)g(x/y)\frac{dy}{y}.$$
\end{definition}



The Mellin transform of a convolution is the product of the Mellin transforms.
\begin{theorem}\label{MellinConvolutionTransform}
Let $f$ and $g$ be functions from $\mathbb{R}_{>0}$ to $\mathbb{C}$. Then
$$\mathcal{M}(f\ast g)(s) = \mathcal{M}(f)(s)\mathcal{M}(g)(s).$$
\end{theorem}



\begin{proof}
\uses{MellinTransform}
This is a straightforward calculation.
\end{proof}



Let $\psi$ be a bumpfunction.
\begin{theorem}\label{SmoothExistence}
There exists a smooth (once differentiable would be enough), nonnegative ``bumpfunction'' $\psi$,
 supported in $[1/2,2]$ with total mass one:
$$
\int_0^\infty \psi(x)\frac{dx}{x} = 1.
$$
\end{theorem}
\begin{proof}
\uses{smooth-ury}
Same idea as Urysohn-type argument.
\end{proof}



The $\psi$ function has Mellin transform $\mathcal{M}(\psi)(s)$ which is entire and decays (at least) like $1/|s|$.
\begin{theorem}\label{MellinOfPsi}
The Mellin transform of $\psi$ is
$$\mathcal{M}(\psi)(s) =  O\left(\frac{1}{|s|}\right),$$
as $|s|\to\infty$.
\end{theorem}

[Of course it decays faster than any power of $|s|$, but it turns out that we will just need one power.]



\begin{proof}
\uses{MellinTransform, SmoothExistence}
Integrate by parts once.
\end{proof}



We can make a delta spike out of this bumpfunction, as follows.
\begin{definition}\label{DeltaSpike}
\uses{SmoothExistence}
Let $\psi$ be a bumpfunction supported in $[1/2,2]$. Then for any $\epsilon>0$, we define the delta spike $\psi_\epsilon$ to be the function from $\mathbb{R}_{>0}$ to $\mathbb{C}$ defined by
$$\psi_\epsilon(x) = \frac{1}{\epsilon}\psi\left(x^{\frac{1}{\epsilon}}\right).$$
\end{definition}

This spike still has mass one:
\begin{lemma}\label{DeltaSpikeMass}
For any $\epsilon>0$, we have
$$\int_0^\infty \psi_\epsilon(x)\frac{dx}{x} = 1.$$
\end{lemma}



\begin{proof}
\uses{DeltaSpike}
Substitute $y=x^{1/\epsilon}$, and use the fact that $\psi$ has mass one, and that $dx/x$ is Haar measure.
\end{proof}



The Mellin transform of the delta spike is easy to compute.
\begin{theorem}\label{MellinOfDeltaSpike}
For any $\epsilon>0$, the Mellin transform of $\psi_\epsilon$ is
$$\mathcal{M}(\psi_\epsilon)(s) = \mathcal{M}(\psi)\left(\epsilon s\right).$$
\end{theorem}



\begin{proof}
\uses{DeltaSpike, MellinTransform}
Substitute $y=x^{1/\epsilon}$, use Haar measure; direct calculation.
\end{proof}



In particular, for $s=1$, we have that the Mellin transform of $\psi_\epsilon$ is $1+O(\epsilon)$.
\begin{corollary}\label{MellinOfDeltaSpikeAt1}
For any $\epsilon>0$, we have
$$\mathcal{M}(\psi_\epsilon)(1) =
\mathcal{M}(\psi)(\epsilon)= 1+O(\epsilon).$$
\end{corollary}



\begin{proof}
\uses{MellinOfDeltaSpike, DeltaSpikeMass}
This is immediate from the above theorem, the fact that $\mathcal{M}(\psi)(0)=1$ (total mass one),
and that $\psi$ is Lipschitz.
\end{proof}



Let $1_{(0,1]}$ be the function from $\mathbb{R}_{>0}$ to $\mathbb{C}$ defined by
$$1_{(0,1]}(x) = \begin{cases}
1 & \text{ if }x\leq 1\\
0 & \text{ if }x>1
\end{cases}.$$
This has Mellin transform
\begin{theorem}\label{MellinOf1}
The Mellin transform of $1_{(0,1]}$ is
$$\mathcal{M}(1_{(0,1]})(s) = \frac{1}{s}.$$
\end{theorem}
[Note: this already exists in mathlib]



What will be essential for us is properties of the smooth version of $1_{(0,1]}$, obtained as the
 Mellin convolution of $1_{(0,1]}$ with $\psi_\epsilon$.
\begin{definition}\label{Smooth1}\uses{MellinOf1, MellinConvolution}
Let $\epsilon>0$. Then we define the smooth function $\widetilde{1_{\epsilon}}$ from $\mathbb{R}_{>0}$ to $\mathbb{C}$ by
$$\widetilde{1_{\epsilon}} = 1_{(0,1]}\ast\psi_\epsilon.$$
\end{definition}



In particular, we have the following
\begin{lemma}\label{Smooth1Properties}
Fix $\epsilon>0$. There is an absolute constant $c>0$ so that:

(1) If $x\leq (1-c\epsilon)$, then
$$\widetilde{1_{\epsilon}}(x) = 1.$$

And (2):
if $x\geq (1+c\epsilon)$, then
$$\widetilde{1_{\epsilon}}(x) = 0.$$
\end{lemma}



\begin{proof}
\uses{Smooth1, MellinConvolution}
This is a straightforward calculation, using the fact that $\psi_\epsilon$ is supported in $[1/2^\epsilon,2^\epsilon]$.
\end{proof}



Combining the above, we have the following Main Lemma of this section on the Mellin transform of $\widetilde{1_{\epsilon}}$.
\begin{lemma}\label{MellinOfSmooth1}\uses{Smooth1Properties, MellinConvolutionTransform, MellinOfDeltaSpikeAt1}
Fix  $\epsilon>0$. Then the Mellin transform of $\widetilde{1_{\epsilon}}$ is
$$\mathcal{M}(\widetilde{1_{\epsilon}})(s) = \frac{1}{s}\left(\mathcal{M}(\psi)\left(\epsilon s\right)\right).$$

For any $s$, we have the bound
$$\mathcal{M}(\widetilde{1_{\epsilon}})(s) = O\left(\frac{1}{\epsilon|s|^2}\right).$$

At $s=1$, we have
$$\mathcal{M}(\widetilde{1_{\epsilon}})(1) = (1+O(\epsilon)).$$
\end{lemma}


