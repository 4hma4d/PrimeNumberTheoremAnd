
In this section, we define the Mellin transform (already in Mathlib, thanks to David Loeffler),
prove its inversion formula, and
derive a number of important properties of some special functions and bumpfunctions.

Def: (Already in Mathlib)
Let $f$ be a function from $\mathbb{R}_{>0}$ to $\mathbb{C}$. We define the Mellin transform of
$f$ to be the function $\mathcal{M}(f)$ from $\mathbb{C}$ to $\mathbb{C}$ defined by
$$\mathcal{M}(f)(s) = \int_0^\infty f(x)x^{s-1}dx.$$

[Note: My preferred way to think about this is that we are integrating over the multiplicative
group $\mathbb{R}_{>0}$, multiplying by a (not necessarily unitary!) character $|\cdot|^s$, and
integrating with respect to the invariant Haar measure $dx/x$. This is very useful in the kinds
of calculations carried out below. But may be more difficult to formalize as things now stand. So
we might have clunkier calculations, which ``magically'' turn out just right - of course they're
explained by the aforementioned structure...]



\begin{definition}[MellinTransform]\label{MellinTransform}\lean{MellinTransform}\leanok
Let $f$ be a function from $\mathbb{R}_{>0}$ to $\mathbb{C}$. We define the Mellin transform of
$f$ to be
the function $\mathcal{M}(f)$ from $\mathbb{C}$ to $\mathbb{C}$ defined by
$$\mathcal{M}(f)(s) = \int_0^\infty f(x)x^{s-1}dx.$$
\end{definition}
[Note: already exists in Mathlib, with some good API.]


\begin{definition}[MellinInverseTransform]\label{MellinInverseTransform}
\lean{MellinInverseTransform}\leanok
Let $F$ be a function from $\mathbb{C}$ to $\mathbb{C}$. We define the Mellin inverse transform of
$F$ to be the function $\mathcal{M}^{-1}(F)$ from $\mathbb{R}_{>0}$ to $\mathbb{C}$ defined by
$$\mathcal{M}^{-1}(F)(x) = \frac{1}{2\pi i}\int_{(\sigma)}F(s)x^{-s}ds,$$
where $\sigma$ is sufficiently large (say $\sigma>2$).
\end{definition}


\begin{lemma}[PerronInverseMellin_lt]\label{PerronInverseMellin_lt}\lean{PerronInverseMellin_lt}
\leanok
Let $0 < t < x$ and $\sigma>0$. Then the inverse Mellin transform of the Perron function
$$F: s\mapsto t^s/s(s+1)$$ is equal to
$$\frac{1}{2\pi i}\int_{(\sigma)}\frac{t^s}{s(s+1)}x^{-s}ds
= 0.$$
\end{lemma}


\begin{proof}
\uses{Perron.formulaLtOne}
\leanok
This is a straightforward calculation.
\end{proof}


\begin{lemma}[PerronInverseMellin_gt]\label{PerronInverseMellin_gt}\lean{PerronInverseMellin_gt}
\leanok
Let $0 < x < t$ and $\sigma>0$. Then the inverse Mellin transform of the Perron function is equal
to
$$\frac{1}{2\pi i}\int_{(\sigma)}\frac{t^s}{s(s+1)}x^{-s}ds = 1 - x / t.$$
\end{lemma}


\begin{proof}
\uses{Perron.formulaGtOne}\leanok
This is a straightforward calculation.
\end{proof}


\begin{lemma}[PartialIntegration]\label{PartialIntegration}\lean{PartialIntegration}\leanok
Let $f, g$ be once differentiable functions from $\mathbb{R}_{>0}$ to $\mathbb{C}$ so that $fg'$
and $f'g$ are both integrable, and $f*g (x)\to 0$ as $x\to 0^+,\infty$.
Then
$$
\int_0^\infty f(x)g'(x) dx = -\int_0^\infty f'(x)g(x)dx.
$$
\end{lemma}


\begin{proof}\leanok
Partial integration.
\end{proof}


\begin{theorem}[MellinInversion]\label{MellinInversion}\lean{MellinInversion}\leanok
Let $f$ be a twice differentiable function from $\mathbb{R}_{>0}$ to $\mathbb{C}$, and
let $\sigma$
be sufficiently large. Then
$$f(x) = \frac{1}{2\pi i}\int_{(\sigma)}\mathcal{M}(f)(s)x^{-s}ds.$$
\end{theorem}

%[Note: How ``nice''? Schwartz (on $(0,\infty)$) is certainly enough. As we formalize
%this, we can add whatever
% conditions are necessary for the proof to go through.]


\begin{proof}\leanok
\uses{PartialIntegration, formulaLtOne, formulaGtOne, MellinTransform,
MellinInverseTransform, PerronInverseMellin_gt, PerronInverseMellin_lt}
%MellinInversion_aux1, MellinInversion_aux2, MellinInversion_aux3,
%MellinInversion_aux4, }
The proof is from [Goldfeld-Kontorovich 2012].
Integrate by parts twice (assuming $f$ is twice differentiable, and all occurring
integrals converge absolutely, and
boundary terms vanish).
$$
\mathcal{M}(f)(s) = \int_0^\infty f(x)x^{s-1}dx = - \int_0^\infty f'(x)x^s\frac{1}{s}dx
= \int_0^\infty f''(x)x^{s+1}\frac{1}{s(s+1)}dx.
$$
We now have at least quadratic decay in $s$ of the Mellin transform. Inserting this
formula into the inversion formula and Fubini-Tonelli (we now have absolute
convergence!) gives:
$$
RHS = \frac{1}{2\pi i}\left(\int_{(\sigma)}\int_0^\infty
  f''(t)t^{s+1}\frac{1}{s(s+1)}dt\right) x^{-s}ds
$$
$$
= \int_0^\infty f''(t) t \left( \frac{1}{2\pi i}
\int_{(\sigma)}(t/x)^s\frac{1}{s(s+1)}ds\right) dt.
$$
Apply the Perron formula to the inside:
$$
= \int_x^\infty f''(t) t \left(1-\frac{x}{t}\right)dt
= -\int_x^\infty f'(t) dt
= f(x),
$$
where we integrated by parts (undoing the first partial integration), and finally
applied the fundamental theorem of calculus (undoing the second).
\end{proof}


Finally, we need Mellin Convolutions and properties thereof.
\begin{definition}[MellinConvolution]\label{MellinConvolution}\lean{MellinConvolution}
\leanok
Let $f$ and $g$ be functions from $\mathbb{R}_{>0}$ to $\mathbb{C}$. Then we define the
Mellin convolution of $f$ and $g$ to be the function $f\ast g$ from $\mathbb{R}_{>0}$
to $\mathbb{C}$ defined by
$$(f\ast g)(x) = \int_0^\infty f(y)g(x/y)\frac{dy}{y}.$$
\end{definition}


Let us start with a simple property of the Mellin convolution.
\begin{lemma}[MellinConvolutionSymmetric]\label{MellinConvolutionSymmetric}
\lean{MellinConvolutionSymmetric}\leanok
Let $f$ and $g$ be functions from $\mathbb{R}_{>0}$ to $\mathbb{C}$, for $x\neq0$,
$$
  (f\ast g)(x)=(g\ast f)(x)
  .
$$
\end{lemma}


\begin{proof}
  \uses{MellinConvolution}
  By Definition \ref{MellinConvolution},
  $$
    (f\ast g)(x) = \int_0^\infty f(y)g(x/y)\frac{dy}{y}
  $$
  in which we change variables to $z=x/y$:
  $$
    (f\ast g)(x) = \int_0^\infty f(x/z)g(z)\frac{dz}{z}
    =(g\ast f)(x)
    .
  $$
\end{proof}


The Mellin transform of a convolution is the product of the Mellin transforms.
\begin{theorem}[MellinConvolutionTransform]\label{MellinConvolutionTransform}
\lean{MellinConvolutionTransform}\leanok
Let $f$ and $g$ be functions from $\mathbb{R}_{>0}$ to $\mathbb{C}$ such that
\begin{equation}
  (x,y)\mapsto f(y)\frac{g(x/y)}yx^{s-1}
  \label{eq:assm_integrable_Mconv}
\end{equation}
is absolutely integrable on $[0,\infty)^2$.
Then
$$\mathcal{M}(f\ast g)(s) = \mathcal{M}(f)(s)\mathcal{M}(g)(s).$$
\end{theorem}


\begin{proof}
\uses{MellinTransform,MellinConvolution}
By Definitions \ref{MellinTransform} and \ref{MellinConvolution}
$$
  \mathcal M(f\ast g)(s)=
  \int_0^\infty \int_0^\infty f(y)g(x/y)x^{s-1}\frac{dy}ydx
$$
By (\ref{eq:assm_integrable_Mconv}) and Fubini's theorem,
$$
  \mathcal M(f\ast g)(s)=
  \int_0^\infty \int_0^\infty f(y)g(x/y)x^{s-1}dx\frac{dy}y
$$
in which we change variables from $x$ to $z=x/y$:
$$
  \mathcal M(f\ast g)(s)=
  \int_0^\infty \int_0^\infty f(y)g(z)y^{s-1}z^{s-1}dzdy
$$
which, by Definition \ref{MellinTransform}, is
$$
  \mathcal M(f\ast g)(s)=
  \mathcal M(f)(s)\mathcal M(g)(s)
  .
$$

\end{proof}


Let $\psi$ be a bumpfunction.
\begin{theorem}[SmoothExistence]\label{SmoothExistence}\lean{SmoothExistence}\leanok
There exists a smooth (once differentiable would be enough), nonnegative ``bumpfunction'' $\psi$,
 supported in $[1/2,2]$ with total mass one:
$$
\int_0^\infty \psi(x)\frac{dx}{x} = 1.
$$
\end{theorem}


\begin{proof}\leanok
\uses{smooth-ury}
Same idea as Urysohn-type argument.
\end{proof}


The $\psi$ function has Mellin transform $\mathcal{M}(\psi)(s)$ which is entire and decays (at
least) like $1/|s|$.
\begin{theorem}[MellinOfPsi]\label{MellinOfPsi}\lean{MellinOfPsi}\leanok
The Mellin transform of $\psi$ is
$$\mathcal{M}(\psi)(s) =  O\left(\frac{1}{|s|}\right),$$
as $|s|\to\infty$ with $\sigma_1 \le \Re(s) \le \sigma_2$.
\end{theorem}

[Of course it decays faster than any power of $|s|$, but it turns out that we will just need one
power.]


\begin{proof}
\uses{MellinTransform, SmoothExistence}
Integrate by parts:
$$
\left|\int_0^\infty \psi(x)x^s\frac{dx}{x}\right| =
\left|-\int_0^\infty \psi'(x)\frac{x^{s}}sdx\right|
$$
$$
\le \frac{1}{|s|} \int_{1/2}^2|\psi'(x)|x^{\Re(s)}dx.
$$
Since $\Re(s)$ is bounded, the right-hand side is bounded by a
constant times $1/|s|$.
\end{proof}


We can make a delta spike out of this bumpfunction, as follows.
\begin{definition}[DeltaSpike]\label{DeltaSpike}\lean{DeltaSpike}\leanok
\uses{SmoothExistence}
Let $\psi$ be a bumpfunction supported in $[1/2,2]$. Then for any $\epsilon>0$, we define the
delta spike $\psi_\epsilon$ to be the function from $\mathbb{R}_{>0}$ to $\mathbb{C}$ defined by
$$\psi_\epsilon(x) = \frac{1}{\epsilon}\psi\left(x^{\frac{1}{\epsilon}}\right).$$
\end{definition}


This spike still has mass one:
\begin{lemma}[DeltaSpikeMass]\label{DeltaSpikeMass}\lean{DeltaSpikeMass}\leanok
For any $\epsilon>0$, we have
$$\int_0^\infty \psi_\epsilon(x)\frac{dx}{x} = 1.$$
\end{lemma}


\begin{proof}\leanok
\uses{DeltaSpike}
Substitute $y=x^{1/\epsilon}$, and use the fact that $\psi$ has mass one, and that $dx/x$ is Haar
measure.
\end{proof}


The Mellin transform of the delta spike is easy to compute.
\begin{theorem}[MellinOfDeltaSpike]\label{MellinOfDeltaSpike}\lean{MellinOfDeltaSpike}\leanok
For any $\epsilon>0$, the Mellin transform of $\psi_\epsilon$ is
$$\mathcal{M}(\psi_\epsilon)(s) = \mathcal{M}(\psi)\left(\epsilon s\right).$$
\end{theorem}


\begin{proof}\leanok
\uses{DeltaSpike, MellinTransform}
Substitute $y=x^{1/\epsilon}$, use Haar measure; direct calculation.
\end{proof}


In particular, for $s=1$, we have that the Mellin transform of $\psi_\epsilon$ is $1+O(\epsilon)$.
\begin{corollary}[MellinOfDeltaSpikeAt1]\label{MellinOfDeltaSpikeAt1}\lean{MellinOfDeltaSpikeAt1}
\leanok
For any $\epsilon>0$, we have
$$\mathcal{M}(\psi_\epsilon)(1) =
\mathcal{M}(\psi)(\epsilon).$$
\end{corollary}


\begin{proof}\leanok
\uses{MellinOfDeltaSpike, DeltaSpikeMass}
This is immediate from the above theorem.
\end{proof}


\begin{lemma}[MellinOfDeltaSpikeAt1_asymp]\label{MellinOfDeltaSpikeAt1_asymp}
\lean{MellinOfDeltaSpikeAt1_asymp}\leanok
As $\epsilon\to 0$, we have
$$\mathcal{M}(\psi_\epsilon)(1) = 1+O(\epsilon).$$
\end{lemma}


\begin{proof}
\uses{MellinTransform,MellinOfDeltaSpikeAt1,SmoothExistence}
By Lemma \ref{MellinOfDeltaSpikeAt1},
$$
  \mathcal M(\psi_\epsilon)(1)=\mathcal M(\psi)(\epsilon)
$$
which by Definition \ref{MellinTransform} is
$$
  \mathcal M(\psi)(\epsilon)=\int_0^\infty\psi(x)x^{\epsilon-1}dx
  .
$$
Since $\psi(x) x^{\epsilon-1}$ is integrable (because $\psi$ is continuous and compactly supported),
$$
  \mathcal M(\psi)(\epsilon)-\int_0^\infty\psi(x)\frac{dx}x=\int_0^\infty\psi(x)(x^{\epsilon-1}-x^{-1})dx
  .
$$
By Taylor's theorem,
$$
  x^{\epsilon-1}-x^{-1}=O(\epsilon)
$$
so, since $\psi$ is absolutely integrable,
$$
  \mathcal M(\psi)(\epsilon)-\int_0^\infty\psi(x)\frac{dx}x=O(\epsilon)
  .
$$
We conclude the proof using Theorem \ref{SmoothExistence}.
\end{proof}


Let $1_{(0,1]}$ be the function from $\mathbb{R}_{>0}$ to $\mathbb{C}$ defined by
$$1_{(0,1]}(x) = \begin{cases}
1 & \text{ if }x\leq 1\\
0 & \text{ if }x>1
\end{cases}.$$
This has Mellin transform
\begin{theorem}[MellinOf1]\label{MellinOf1}\lean{MellinOf1}\leanok
The Mellin transform of $1_{(0,1]}$ is
$$\mathcal{M}(1_{(0,1]})(s) = \frac{1}{s}.$$
\end{theorem}
[Note: this already exists in mathlib]


\begin{proof}\leanok
\uses{MellinTransform}
This is a straightforward calculation.
\end{proof}


What will be essential for us is properties of the smooth version of $1_{(0,1]}$, obtained as the
 Mellin convolution of $1_{(0,1]}$ with $\psi_\epsilon$.
\begin{definition}[Smooth1]\label{Smooth1}\lean{Smooth1}
\uses{MellinOf1, MellinConvolution}\leanok
Let $\epsilon>0$. Then we define the smooth function $\widetilde{1_{\epsilon}}$ from
$\mathbb{R}_{>0}$ to $\mathbb{C}$ by
$$\widetilde{1_{\epsilon}} = 1_{(0,1]}\ast\psi_\epsilon.$$
\end{definition}


\begin{lemma}[Smooth1Properties_estimate]\label{Smooth1Properties_estimate}
\lean{Smooth1Properties_estimate}\leanok
For $\epsilon>0$,
$$
  \log2>\frac{1-2^{-\epsilon}}\epsilon
$$
\end{lemma}


\begin{proof}
Let $\alpha:=2^\epsilon>1$, in terms of which we wish to prove
$$
  1\geqslant\frac{1-1/\alpha}{\log \alpha}
  .
$$
In addition,
$$
  \frac d{d\alpha}\left(1-\frac1\alpha-\log \alpha\right)=\frac1{\alpha^2}(1-\alpha)<0
$$
so $1-1/\alpha-\log\alpha$ is monotone decreasing so it is smaller than its value at $\alpha=1$:
$$
  1-\frac1\alpha-\log\alpha<0
  .
$$
We conclude the proof using $\log\alpha>0$.
\end{proof}


In particular, we have the following two properties.
\begin{lemma}[Smooth1Properties_below]\label{Smooth1Properties_below}
\lean{Smooth1Properties_below}\leanok
Fix $\epsilon>0$. There is an absolute constant $c>0$ so that:
If $0<x\leq (1-c\epsilon)$, then
$$\widetilde{1_{\epsilon}}(x) = 1.$$
\end{lemma}


\begin{proof}
\uses{Smooth1, MellinConvolution,DeltaSpikeMass, Smooth1Properties_estimate}
Opening the definition, we have that the Mellin convolution of $1_{(0,1]}$ with $\psi_\epsilon$ is
$$
\int_0^\infty 1_{(0,1]}(y)\psi_\epsilon(x/y)\frac{dy}{y}
=
\int_0^1 \psi_\epsilon(x/y)\frac{dy}{y}.
$$
The support of $\psi_\epsilon$ is contained in $[1/2^\epsilon,2^\epsilon]$, so
$y \in [1/2^\epsilon x,2^\epsilon x]$. If $x \le 2^{-\epsilon}$, then the integral is the same as that over $(0,\infty)$:
$$
\int_0^\infty 1_{(0,1]}(y)\psi_\epsilon(x/y)\frac{dy}{y}
=
\int_0^\infty \psi_\epsilon(x/y)\frac{dy}{y}.
$$
in which we change variables to $z=x/y$ (using $x>0$):
$$
\int_0^\infty 1_{(0,1]}(y)\psi_\epsilon(x/y)\frac{dy}{y}
=
\int_0^\infty \psi_\epsilon(z)\frac{dz}{z}.
$$
which is equal to one by Lemma \ref{DeltaSpikeMass}.
We then choose
$$
  c:=\log 2
$$
which satisfies
$$
  c\geqslant\frac{1-2^{-\epsilon}}\epsilon
$$
by Lemma \ref{Smooth1Properties_estimate}, so
$$
  1-c\epsilon\leqslant 2^{-\epsilon}
  .
$$
\end{proof}


\begin{lemma}[Smooth1Properties_above]\label{Smooth1Properties_above}
\lean{Smooth1Properties_above}\leanok
Fix $0<\epsilon<1$. There is an absolute constant $c>0$ so that:
if $x\geq (1+c\epsilon)$, then
$$\widetilde{1_{\epsilon}}(x) = 0.$$
\end{lemma}


\begin{proof}
\uses{Smooth1, MellinConvolution, Smooth1Properties_estimate}
Again the Mellin convolution is
$$\int_0^1 \psi_\epsilon(x/y)\frac{dy}{y},$$
but now if $x \ge 2^\epsilon$, then the support of $\psi_\epsilon$ is disjoint
from the region of integration, and hence the integral is zero.
We choose
$$
  c:=2\log 2
  .
$$
By Lemma \ref{Smooth1Properties_estimate},
$$
  c\geqslant 2\frac{1-2^{-\epsilon}}\epsilon\geqslant 2^\epsilon\frac{1-2^{-\epsilon}}\epsilon
  =
  \frac{2^\epsilon-1}\epsilon
$$
so
$$
  1+c\epsilon\geqslant 2^\epsilon
  .
$$
\end{proof}


\begin{lemma}[Smooth1Nonneg]\label{Smooth1Nonneg}\lean{Smooth1Nonneg}\leanok
If $\psi$ is nonnegative, then $\widetilde{1_{\epsilon}}(x)$ is nonnegative.
\end{lemma}


\begin{proof}\uses{Smooth1,MellinConvolution,DeltaSpike}
By Definitions \ref{Smooth1}, \ref{MellinConvolution} and \ref{DeltaSpike}
$$
  \widetilde{1_\epsilon}(x)=\int_0^\infty 1_{(0,1]}(y)\frac1\epsilon\psi((x/y)^{\frac1\epsilon}) \frac{dy}y
$$
and all the factors in the integrand are nonnegative.
\end{proof}


\begin{lemma}[Smooth1LeOne]\label{Smooth1LeOne}\lean{Smooth1LeOne}\leanok
If $\psi$ is nonnegative and has mass one, then $\widetilde{1_{\epsilon}}(x)\le 1$, $\forall x>0$.
\end{lemma}


\begin{proof}\uses{Smooth1,MellinConvolution,DeltaSpike,SmoothExistence}
By Definitions \ref{Smooth1}, \ref{MellinConvolution} and \ref{DeltaSpike}
$$
  \widetilde{1_\epsilon}(x)=\int_0^\infty 1_{(0,1]}(y)\frac1\epsilon\psi((x/y)^{\frac1\epsilon}) \frac{dy}y
$$
and since $1_{(0,1]}(y)\le 1$, and all the factors in the integrand are nonnegative,
$$
  \widetilde{1_\epsilon}(x)\le\int_0^\infty \frac1\epsilon\psi((x/y)^{\frac1\epsilon}) \frac{dy}y
$$
in which we change variables to $z=(x/y)^{\frac1\epsilon}$:
$$
  \widetilde{1_\epsilon}(x)\le\int_0^\infty \psi(z) \frac{dz}z
$$
which by Theorem \ref{SmoothExistence} is 1.
\end{proof}


Combining the above, we have the following three Main Lemmata of this section on the Mellin
transform of $\widetilde{1_{\epsilon}}$.
\begin{lemma}[MellinOfSmooth1a]\label{MellinOfSmooth1a}\lean{MellinOfSmooth1a}\leanok
Fix  $\epsilon>0$. Then the Mellin transform of $\widetilde{1_{\epsilon}}$ is
$$\mathcal{M}(\widetilde{1_{\epsilon}})(s) =
\frac{1}{s}\left(\mathcal{M}(\psi)\left(\epsilon s\right)\right).$$
\end{lemma}


\begin{proof}\uses{Smooth1,MellinConvolutionTransform, MellinOfDeltaSpike, MellinOf1, MellinConvolutionSymmetric}
By Definition \ref{Smooth1},
$$
  \mathcal M(\widetilde{1_\epsilon})(s)
  =\mathcal M(1_{(0,1]}\ast\psi_\epsilon)(s)
  .
$$
We wish to apply Theorem \ref{MellinConvolutionTransform}.
To do so, we must prove that
$$
  (x,y)\mapsto 1_{(0,1]}(y)\psi_\epsilon(x/y)/y
$$
is integrable on $[0,\infty)^2$.
It is actually easier to do this for the convolution: $\psi_\epsilon\ast 1_{(0,1]}$, so we use Lemma \ref{MellinConvolutionSymmetric}: for $x\neq0$,
$$
  1_{(0,1]}\ast\psi_\epsilon(x)=\psi_\epsilon\ast 1_{(0,1]}(x)
  .
$$
Now, for $x=0$, both sides of the equation are 0, so the equation also holds for $x=0$.
Therefore,
$$
  \mathcal M(\widetilde{1_\epsilon})(s)
  =\mathcal M(\psi_\epsilon\ast 1_{(0,1]})(s)
  .
$$
Now,
$$
  (x,y)\mapsto \psi_\epsilon(y)1_{(0,1]}(x/y)\frac{x^{s-1}}y
$$
has compact support that is bounded away from $y=0$ (specifically $y\in[2^{-\epsilon},2^\epsilon]$ and $x\in(0,y]$), so it is integrable.
We can thus apply Theorem \ref{MellinConvolutionTransform} and find
$$
  \mathcal M(\widetilde{1_\epsilon})(s)
  =\mathcal M(\psi_\epsilon)(s)\mathcal M(1_{(0,1]})(s)
  .
$$
By Lemmas \ref{MellinOf1} and \ref{MellinOfDeltaSpike},
$$
  \mathcal M(\widetilde{1_\epsilon})(s)
  =\frac1s\mathcal M(\psi)(\epsilon s)
  .
$$
\end{proof}


\begin{lemma}[MellinOfSmooth1b]\label{MellinOfSmooth1b}\lean{MellinOfSmooth1b}\leanok
Given $0<\sigma_1\le\sigma_2$, for any $s$ such that $\sigma_1\le\mathcal Re(s)\le\sigma_2$, we have
$$\mathcal{M}(\widetilde{1_{\epsilon}})(s) = O\left(\frac{1}{\epsilon|s|^2}\right).$$
\end{lemma}


\begin{proof}\uses{MellinOfSmooth1a, MellinOfPsi}
Use Lemma \ref{MellinOfSmooth1a} and the bound in Lemma \ref{MellinOfPsi}.
\end{proof}


\begin{lemma}[MellinOfSmooth1c]\label{MellinOfSmooth1c}\lean{MellinOfSmooth1c}\leanok
At $s=1$, we have
$$\mathcal{M}(\widetilde{1_{\epsilon}})(1) = (1+O(\epsilon)).$$
\end{lemma}


\begin{proof}\uses{MellinOfSmooth1a, MellinOfDeltaSpikeAt1, MellinOfDeltaSpikeAt1_asymp}
Follows from Lemmas \ref{MellinOfSmooth1a}, \ref{MellinOfDeltaSpikeAt1} and \ref{MellinOfDeltaSpikeAt1_asymp}.
\end{proof}

